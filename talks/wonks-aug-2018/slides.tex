\documentclass[12pt]{beamer}
%\usecolortheme{seagull}
%\usecolortheme{wolverine} yuk
%\usecolortheme{beetle}
\usecolortheme{dove} % black on white
\usepackage[T1]{fontenc}
\usepackage{garamond}
\usefonttheme{serif}
\usepackage{multicol}
\usepackage{pifont}
\usepackage{etex}
\usepackage{graphicx}
\usepackage{amsmath}
\usepackage{amsthm}
\usepackage{amssymb}
\usepackage{semantic}
\usepackage[all]{xy}
\usepackage{color}
\usepackage{listings}
\usepackage{fancybox}
\usepackage{stmaryrd}
\usepackage{rotating}
\usepackage{wasysym}
\usepackage{ulem}

\definecolor{darkgreen}{rgb}{0.0, 0.6, 0.3}

\usepackage{enumitem}
\setitemize{label=\usebeamerfont*{itemize item}%
  \usebeamercolor[fg]{itemize item}
  \usebeamertemplate{itemize item}}
  \setlist{itemsep=1ex}



\newcommand{\Gbox}[1]{\colorbox{lightgray}{#1}}
\newcommand{\Rbox}[1]{\colorbox{pink}{#1}}

\newcommand{\featstart}{\hfill}
\newcommand{\featend}{\hfill\hfill}
\newcommand{\feat}[1]{{\featstart#1\featend}}

\newcommand{\Topcircle}{\begin{turn}{270}\Leftcircle\end{turn}}
\newcommand{\BOTTOMCIRCLE}{\begin{turn}{270}\RIGHTCIRCLE\end{turn}}
\newcommand{\halfcircle}{\parbox{0in}{\Topcircle}\parbox{1.65ex}{\BOTTOMCIRCLE}{}}

\newcommand{\featY}{\feat{\CIRCLE}} % Has feature fully
\newcommand{\featP}{\feat{\halfcircle}} % Has feature partially
\newcommand{\featN}{\feat{\Circle}} % Does not have feature


\newcommand{\labeltag}[1]{\label{#1}\tag{\textsc{#1}}}
\newcommand{\type}{\vdash}
\newcommand{\typeS}{\vdash_{STLC}}
\newcommand{\typeG}{\vdash}
\newcommand{\typeCC}{\vdash_{C}}

\newcommand{\evall}{\Downarrow }
\newcommand{\evallS}{\Downarrow_{STLC} }
\newcommand{\evallG}{\Downarrow}
\newcommand{\evallCC}{\Downarrow_{C}}
\newcommand{\evallD}{\Downarrow_{DTLC}}

\newcommand{\reduce}{\longrightarrow}
\newcommand{\becomes}{\longrightarrow}

%\newcommand{\EE}{{\cal E}}
%\newcommand{\FF}{{\cal F}}
\newcommand{\Hole}{\Box}

\newcommand{\divergeG}{\Uparrow}
\newcommand{\subtype}{<:}
\newcommand{\consis}{\sim}

\newcommand{\embed}[1]{\lceil #1 \rceil}
\newcommand{\bl}[1]{{\color{blue} #1}}
\newcommand{\rd}[1]{{\color{red} #1}}
\newcommand{\pr}[1]{{\color{purple} #1}}
\newcommand{\kw}[1]{\mathtt{#1}}

\newcommand{\labels}[1]{\mathit{labels}(#1)}
\newcommand{\static}[2]{\mathit{static}(#1,#2)}
\newcommand{\safe}[1]{\mathrel{\mathit{safe}} #1}
\newcommand{\lo}[1]{\overline{#1}}
\newcommand{\rng}[1]{\mathit{rng}(#1)}

\newcommand{\semi}{\mathbin{;}}
\newcommand{\id}{\key{id}}
\newcommand{\Id}[1]{\id_{#1}}
\newcommand{\fail}[3]{\bot^{#1}_{#2 \Rightarrow #3}}
\newcommand{\Fail}[1]{\bot^{#1}}
\newcommand{\FAIL}[3]{\bot^{#2}}
\newcommand{\qu}[2]{{{#2}\query^{#1}}}
\newcommand{\pl}[1]{{#1\pling}}
\newcommand{\query}{\mathtt{?}}
\newcommand{\pling}{\mathtt{!}}

\newcommand{\bcfun}[1]{\langle\!\langle #1 \rangle\!\rangle}
\newcommand{\MergeT}{\sqcap}
\newcommand{\RefC}[1]{\key{Ref}(#1)}
\newcommand{\error}{\key{error}}


\newcommand{\Obj}{\key{Obj}}
\newcommand{\String}{\key{String}}
\newcommand{\Double}{\key{Double}}

%\newcommand{\If}[3]{\key{if}\,#1\key{if}\,#2\key{if}#3}


\newcommand{\ba}{\begin{array}}
\newcommand{\ea}{\end{array}}
\newenvironment{stack}{\ba{@{}l@{}}}{\ea}
\newenvironment{branch}{\left\{\ba{@{}l@{\qquad}l@{}}}{\ea\right\}}
\newenvironment{syntax}{\[\ba{l@{\;\;}lcl}}{\ea\]}
\newcommand{\dotspace}{.\,}
\newcommand{\key}[1]{\ensuremath{\mathtt{#1}}}
\newcommand{\Base}{B}
\newcommand{\dyn}{\star}
\newcommand{\Dyn}{\ensuremath{\star}}
\newcommand{\Int}{\key{Int}}
\newcommand{\Float}{\key{float}}
\newcommand{\Bool}{\key{Bool}}
\newcommand{\Str}{\key{String}}
\newcommand{\Ref}{\key{Ref}\,}
\newcommand{\tapp}{\;\,}
\newcommand{\tu}{{\to}}
\newcommand{\To}{\Rightarrow}
\newcommand{\Let}{\key{let}\;}
\newcommand{\Letrec}{\key{let}\,\key{rec}\;}
\newcommand{\In}{\key{in}\;}
\newcommand{\If}{\mathsf{if}\;}
\newcommand{\Then}{\;\mathsf{then}\;}
\newcommand{\Else}{\mathsf{else}\;}
\newcommand{\True}{\key{true}}
\newcommand{\False}{\key{false}}
\newcommand{\as}{\mathrel{\key{as}}}
\newcommand{\op}{\mathit{op}}
\newcommand{\dom}[1]{\mathit{dom}(#1)}
\newcommand{\cod}[1]{\mathit{cod}(#1)}
\newcommand{\blame}[1]{\key{blame}\,#1}
\newcommand{\pblame}[2]{\key{blame}\,#1@#2}
\newcommand{\ledyn}{\sqsubseteq}
\newcommand{\IS}{\mathrel{\mathtt{is}}}
\newcommand{\cast}[1]{\overset{#1}{\Rightarrow}}
%\newcommand{\mkcast}[1]{\langle\!\langle#1\rangle\!\rangle}
\newcommand{\mkcast}[1]{(#1)}
\newcommand{\alloc}{\key{ref}\,}
\newcommand{\deref}{\texttt{!}}
\newcommand{\update}{\mathrel{\texttt{:=}}}
\newcommand{\all}[1]{\forall #1.\,}
\newcommand{\ftv}[1]{\mathrm{ftv}(#1)}
\newcommand{\CAST}[1]{\langle #1 \rangle}
\newcommand{\new}[1]{\nu #1.\;}
\newcommand{\case}[3]{\key{case}\,#1\,\key{of}\,\key{inl}\,x\Rightarrow\,#2\,| \,\key{inr}\,x\Rightarrow \,#3}
\newcommand{\join}[2]{#1 \sqcup #2 }
\newcommand{\meet}[2]{#1 \sqcap #2 }

\newcommand{\EE}[2]{\mathcal{E}\llbracket #1 \rrbracket #2}
\newcommand{\FF}[3]{\mathcal{F}(#1, #2, #3)}
\newcommand{\CC}[1]{\mathcal{C}\llbracket #1 \rrbracket}

\newcommand{\EXP}[1]{\,#1\,}
\newcommand{\CE}[0]{\mathcal{E}}
\newcommand{\SEM}[1]{\llbracket #1 \rrbracket}
\newcommand{\TSEM}[1]{\mathcal{T}\llbracket #1 \rrbracket}
\newcommand{\lam}[1]{\lambda #1.\,}
\newcommand{\LAM}[1]{\lambda\!\!\!\lambda #1.\,}
\newcommand{\of}[0]{{:}}
\newcommand{\by}[0]{{:=}}
\newcommand{\INT}[0]{\mathtt{int}}
\newcommand{\REF}[1]{\mathtt{ref}\,#1}
\newcommand{\fix}[1]{\mathtt{fix}\,#1.\,}
\newcommand{\FST}[1]{\pi_1\, #1}
\newcommand{\SND}[1]{\pi_2\, #1}
\newcommand{\fst}[1]{\mathsf{fst}(#1)}
\newcommand{\snd}[1]{\mathsf{snd}(#1)}
\newcommand{\emptyenv}[0]{\mathsf{empty}}
\newcommand{\ext}[3]{\mathsf{extend}(#1,#2,#3)}
\newcommand{\tyext}[1]{\mathsf{tyExtend}(#1)}
\newcommand{\lookup}[2]{\mathsf{lookup}(#1,#2)}
\newcommand{\shift}[3]{\,\uparrow^{#1}_{#2}(#3)}
\newcommand{\app}[0]{\;}
\newcommand{\LET}[0]{\mathbf{let}\,}
\newcommand{\IN}[0]{\,\mathbf{in}\,}
\newcommand{\abs}[1]{\mathsf{thunk}(#1)}
\newcommand{\val}[1]{\overline{#1}}
\newcommand{\CASE}[0]{\textsf{case}\,}
\newcommand{\OF}[0]{\,\textsf{of}\,}
%% \newcommand{\IF}[0]{\textbf{if}\,}
%% \newcommand{\THEN}[0]{\,\textbf{then}\,}
%% \newcommand{\ELSE}[0]{\,\textbf{else}\,}
\newcommand{\IF}[0]{\mathsf{ifz}(}
\newcommand{\THEN}[0]{,}
\newcommand{\ELSE}[0]{,}
\newcommand{\FI}[0]{)}
\newcommand{\return}[0]{\mathsf{return}\;}
\newcommand{\bind}[0]{\mathsf{bind}\;}
\newcommand{\down}[0]{\mathsf{down}\;}
\newcommand{\wrong}[0]{\mathsf{wrong}}
\newcommand{\plus}[0]{\mathrel{\mathsf{plus}}}
\newcommand{\zero}[0]{\mathsf{zero}}
\newcommand{\SET}[1]{\mathcal{P}(#1)}
\newcommand{\FSET}[1]{\mathcal{P}_f(#1)}
\newcommand{\store}[0]{\mathsf{store}}
\newcommand{\CHOOSE}[0]{\mathsf{choose}\,}

\newcommand*\oldmacro{}%
\let\oldmacro\insertshorttitle%
\renewcommand*\insertshorttitle{%
  \oldmacro\hfill%
  \insertframenumber\,/\,\inserttotalframenumber}

\setbeamertemplate{navigation symbols}{}
\setbeamertemplate{footline}[frame number]

%\newtheorem{definition}{Definition}
\newtheorem{conjecture}[theorem]{\translate{Conjecture}}
\newtheorem{proposition}[theorem]{\translate{Proposition}}

\lstdefinestyle{basic}{
%showstringspaces=false,
language=Python,
columns=fullflexible,
%basicstyle=\sffamily\small,%
basicstyle=\ttfamily,%
%columns=fixed,
%basewidth=0.49em,
%lineskip=0pt,
%escapechar=@,xleftmargin=1pc,%
keywordstyle=\ttfamily,
mathescape=true,%
moredelim=**[is][\color{blue}]{@}{@},
moredelim=[is][\color{red}]{|}{|},
moredelim=[is][\color{blue}]{~}{~},
%commentstyle=\rmfamily,%
%morekeywords={return,fix,var,proc,fun,func},%
%deletekeywords={int,bool}
}
\lstset{style=basic}

\garamond

\title[Back to Denotational]{Back to the Future with \\ Denotational Semantics}
\author{Jeremy G. Siek \\[1ex]
 Indiana University, Bloomington
}
\date{}
%% \institute{\normalsize 
%%  Indiana University, Bloomington
%% }

% 3 hours

%\newcommand\footnotemark{}
%\renewcommand\footnoterule{}
\setbeamercolor{footnote mark}{fg=white}

\begin{document}

\frame{

%% \begin{tabular}{ll} 
%% \begin{minipage}{0.45\textwidth}
\centering
\large Intersection Types, Sub-formula Property, and  \\ 
   the Functional Character of the Lambda Calculus \\[2ex]
\normalsize Jeremy G. Siek \\
Indiana University\\[2ex]
PL Wonks \\
August 2018
%% \end{minipage}
%% &
%% \begin{minipage}{0.45\textwidth}
%% \includegraphics[height=3in]{btf}
%% \end{minipage}
%% \end{tabular}

}
%===============================================================================
\frame{

}

%===============================================================================

\frame{
\frametitle{Outline}
\begin{itemize}
\item ...
\end{itemize}
}


%===============================================================================

\frame{
\frametitle{Tabulating Functions}

Syntax of an applied lambda calculus:
\[
  \begin{array}{rcl}
  && n \in \mathbb{Z} \qquad x \in \mathbb{X} \;\;\text{(program variables)}\\
  e \in \mathbb{E} & ::= & x \mid \lam{x} e \mid e \; e \mid \\
    & & n \mid e \oplus e \mid \IF e \THEN e \ELSE e  \FI\\
  \oplus & ::= & + \mid - \mid \times 
  \end{array}
\]

Values:
\begin{center}
  \fbox{\fbox{
      $
      \begin{array}{ll}
      v ::= n \mid \{ (v_1,v'_1),\ldots,(v_n,v'_n) \}\\[1ex]
      \mathbb{V} = \mathbb{Z} + \mathcal{P}_{f}(\mathbb{V} \times \mathbb{V}) \quad \dagger
      \end{array}
      $
}}
\end{center}

Let $t$, for table, range over $\mathcal{P}_{f}(\mathbb{V} \times \mathbb{V}) $.

}

%===============================================================================
\frame{
\frametitle{A Compositional Big-Step Semantics}
\small
%\vspace{5pt}
\fbox{$\rho \vdash e \Rightarrow v$}
\begin{gather*}
  \inference
      {}
      {\rho \vdash x \Rightarrow \rho(x)}
  \quad
  \inference
      {\forall (v,v')\in t,\\ \rho(x{:=}v) \vdash e \Rightarrow v'}
      {\rho \vdash \lam{x}e \Rightarrow t}
  \quad
  \inference
      {\rho \vdash e_1 \Rightarrow t &
       \rho \vdash e_2 \Rightarrow v \\
      (v,v') \in t}
      {\rho \vdash (e_1\app e_2) \Rightarrow v'}
  \\[2ex]
  \inference{}{\rho \vdash n \Rightarrow n}
  \quad
  \inference
      {\rho \vdash e_1 \Rightarrow n_1 &
       \rho \vdash e_2 \Rightarrow n_2}
      {\rho \vdash e_1 \oplus e_2 \Rightarrow n_1 \oplus n_2}
  \\[2ex]
  \inference{\rho \vdash e_1 \Rightarrow 0 & 
    \rho \vdash e_2 \Rightarrow v}
            {\rho \vdash \IF e_1 \THEN e_2 \ELSE e_3 \FI \Rightarrow v}
  \inference{\rho \vdash e_1 \Rightarrow n & n \neq 0 &
    \rho \vdash e_3 \Rightarrow v}
            {\rho \vdash \IF e_1 \THEN e_2 \ELSE e_3 \FI \Rightarrow v}
\end{gather*}
%% \hrule
%% \hrule
%% \vspace{5pt}
%% Comparison to big-step semantics:
%% \begin{gather*}
%%   \inference{}
%%             {\varrho \vdash \lam{x} e \Rightarrow \langle \lam{x}e,\varrho \rangle}
%%   \qquad
%%   \inference{\varrho \vdash e_1 \Rightarrow \langle \lam{x}e,\varrho' \rangle&
%%              \varrho \vdash e_1 \Rightarrow v_2 \\
%%              \varrho'(x\by v_2) \vdash e \Rightarrow v}
%%             {\varrho \vdash (e_1 \app e_2) \Rightarrow v}
%% \end{gather*}
}
%===============================================================================
\frame{
\frametitle{The Problem with Self-Application}

Consider
\[
  (\lam{f} f \app f) \app (\lam{g} 1)
\]
Clearly, the result should be $1$.

To obtain $\vdash (\lam{f} f \app f) \app (\lam{g} 1) \Rightarrow 1$ we
need some $t_1$ and $t_2$ s.t.
\[
\vdash (\lam{f} f \app f) \Rightarrow t_1 \qquad
\vdash (\lam{g} 1) \Rightarrow t_2 \qquad
(t_2,1) \in t_1
\]
So we need $\{ f\by t_2 \} \vdash f \app f \Rightarrow 1$
and then
\[
  (t_2,1) \in t_2
\]
but that's impossible!

}
%===============================================================================

\frame{
\frametitle{Fixing Self-Application}

We can fix this problem by allowing a kind of subsumption.

\vspace{10pt}
\fbox{$v \sqsubseteq v$}
\[
    \inference{}{n \sqsubseteq n}
    \qquad
    \inference{t_1 \subseteq t_2}
               {t_1 \sqsubseteq t_2}
    %% \inference{\forall (v_1,v'_1) \in t_1, \exists (v_2,v'_2) \in t_2,\\ 
    %%            v_2 \sqsubseteq v_1 \text{ and } v'_1 \sqsubseteq v'_2}
    %%            {t_1 \sqsubseteq t_2}
\]
\vspace{10pt}

We allow arguments to be larger than necessary:$^\dagger$
\footnote{$^\dagger$ Thank you Allan Jeffrey for tweeting this solution!}
\[
\inference{\rho \vdash e_1 \Rightarrow t &
           \rho \vdash e_2 \Rightarrow v_2 \\
           (v,v') \in t & \rd{v \sqsubseteq v_2}}
          {\rho \vdash (e_1\app e_2) \Rightarrow v'}
\]

}

%===============================================================================
\frame{
\frametitle{Subsumption and Environment Weakening}

To prove equivalent wrt. reduction semantics, we need: 

\vspace{10pt}

\begin{proposition}
If $\rho \vdash e \Rightarrow v$, $v' \sqsubseteq v$, 
and $\mathrm{fv}(e) \vdash \rho \sqsubseteq \rho'$,\\
then $\rho' \vdash e \Rightarrow v'$.
\label{prop:change-env-le}
\end{proposition}

\vspace{10pt}

Example: 
\[
  t_0 = \{(1,2) \} \quad\sqsubseteq\quad t_1 = \{(0,1),(1,2)\}
\]
\[
 \emptyset \vdash \lam{x} x \Rightarrow \{(t_1,t_1)\}
 \qquad
 \emptyset \vdash \lam{x} x \Rightarrow \{(t_1,t_0)\}
\]

}
%===============================================================================
\frame{
\frametitle{Simple Semantics in Relational Form (take 2)}
\small


\fbox{$\rho \vdash e \Rightarrow v$}
\begin{gather*}
  \inference{}{\rho \vdash n \Rightarrow n}
  \quad
  \inference
      {\rho \vdash e_1 \Rightarrow n_1 &
       \rho \vdash e_2 \Rightarrow n_2}
      {\rho \vdash e_1 \oplus e_2 \Rightarrow n_1 \oplus n_2}
  \quad
  \inference
      {\rd{v \sqsubseteq \rho(x)}}
      {\rho \vdash x \Rightarrow v}
  \\[3ex]
  \inference
      {\forall (v,v'){\in} t.\; \rho(x{:=}v) \vdash e \Rightarrow v'}
      {\rho \vdash \lam{x}e \Rightarrow t}
  \quad
  \inference
      {\rho \vdash e_1 \Rightarrow t &
       \rho \vdash e_2 \Rightarrow v_2 \\
      (v,v') \in t & \rd{v \sqsubseteq v_2} & \rd{v_3 \sqsubseteq v'}}
      {\rho \vdash (e_1\app e_2) \Rightarrow v_3}
  \\[3ex]
  \inference{\rho \vdash e_1 \Rightarrow n & n \neq 0 &
    \rho \vdash e_2 \Rightarrow v}
            {\rho \vdash \IF e_1 \THEN e_2 \ELSE e_3 \Rightarrow v}
  \quad
  \inference{\rho \vdash e_1 \Rightarrow 0 & 
    \rho \vdash e_3 \Rightarrow v}
            {\rho \vdash \IF e_1 \THEN e_2 \ELSE e_3 \Rightarrow v}
\end{gather*}
}
%===============================================================================
\frame{
\frametitle{Example of a recursive function}

\begin{align*}
  M & \equiv \lam{x} f \app (\lam{v} (x\app x) \app v) \\
  Z & \equiv \lam{f} M \app M  \\
  F & \equiv \lam{n} \IF n \THEN 1 \ELSE n  \times r(n-1) \FI \\
  H & \equiv \lam{r} F \\
  \mathit{fact} & \equiv Z\app H
\end{align*}
Tables:
\begin{align*}
  F_t(n) &= \{ (n,n!) \} \\
%  H_t(n) &= \{ (\emptyset, F_t(0)), (F_t(0), F_t(1)), \ldots, (F_t(n-1), F_t(n))\} \\
  H_t(n) &= \{ (\emptyset, F_t(0)) \} \cup \{ (F_t(n), F_t(n+1)) \mid n \in \mathbb{N} \} \\
  \rd{M_t(0)} &= \rd{\emptyset} \\
  \rd{M_t(n+1)} &= \rd{M_t(n) \cup \{ (M_t(n), F_t(n)) \}} \\
  Z_t(n) &= \{ (H_t(n), F_t(n)) \}
\end{align*}
We have $\emptyset \vdash Z\app H \Rightarrow F_t(n)$ for any $n$,
so $\emptyset \vdash \mathit{fact}\,n \Rightarrow n!$.

}

%===============================================================================

\frame{
\frametitle{Simple Semantics in Denotational Form}

Of course, relations are isomorphic to set-valued functions:
\[
  \SET{(\mathbb{X} \to \mathbb{V}) \times \mathbb{E} \times \mathbb{V}}
  \quad\cong\quad
  \mathbb{E} \to (\mathbb{X} \to \mathbb{V}) \to \SET{\mathbb{V}}
\]

}

%===============================================================================

\frame{
\frametitle{Simple Semantics in Denotational Form}

%% \[
%% v \in \mathbb{V} \qquad \rho \in \mathbb{X} \rightharpoonup \mathbb{V}
%% \]

\fbox{$\SEM{-}: \mathbb{E} \to (\mathbb{X} \to \mathbb{V}) \to \SET{\mathbb{V}}$}
\begin{align*}
\SEM{ n }\rho &= \{ n \} \\
\SEM{ e_1 \oplus e_2 }\rho &= \{  n_1 \oplus n_2 \mid 
   n_1 \in \SEM{ e_1 }\rho \text{ and } n_2 \in \SEM{ e_2 }\rho \} \\
\SEM{ x }\rho &= \{ v \mid v \sqsubseteq \rho(x) \} \\
\SEM{ \lam{x} e }\rho &= 
  \{ t \mid \forall (v,v')\in t,\, v' \in \SEM{ e }\rho(x{:=}v) \} \\
\SEM{ e_1\;e_2 }\rho &= \left\{ v_3 \, \middle| 
   \begin{array}{l}
   \exists t\, v_2\, v\, v'.\, t \in \SEM{ e_1 }\rho,\, v_2 \in \SEM{ e_2 }\rho, \\
   (v, v') \in t, \, v \sqsubseteq v_2,\, v_3 \sqsubseteq v'
   \end{array}
\right\} \\
\SEM{\IF e_1 \THEN e_2 \ELSE e_3 \FI}\rho &=
\left\{ v\, \middle|\,
  \begin{array}{l}
   \exists n.\, n \in \SEM{e_1}\rho, \\
   \quad (n= 0 \implies v \in \SEM{e_2}\rho) \text{ and}\\
   \quad (n\neq 0 \implies v \in \SEM{e_3}\rho)
  \end{array}
  \right\}
\end{align*}

}

%===============================================================================
\frame{
\frametitle{Properties}

\begin{itemize}
\item $\beta_v$-reduction preserves meaning:
\[
\SEM{(\lam{x}e) \app v} = \SEM{e[v/x]}
\]
\item Sound \& complete wrt. op. sem.\\
  If $\SEM{e}\emptyset = \SEM{n}\emptyset$, then $e \Downarrow n$.\\
  If $e \Downarrow n$, then $\SEM{e} \emptyset = \SEM{n}\emptyset$.
\item The simple semantics is a congruence:\\
  For any context $C$, 
  if $\SEM{e} = \SEM{e'}$, then $\SEM{C[e]} = \SEM{C[e']}$.
\item Sound wrt. contextual equivalence: \\
  If $\SEM{e} = \SEM{e'}$,
  then $C[e] \Downarrow$ iff $C[e'] \Downarrow$
  for any closing $C$.
\item The simple semantics is not fully abstract. \\
  Parallel OR provides a counter example.
\end{itemize}


}

%===============================================================================

\frame{
\frametitle{Outline}

\begin{itemize}
\item A simple semantics of $\lambda$-calculus in two
  forms:\\ relational and denotational.
\item \textbf{Plotkin, Scott, and Engeler's models}
\item Correctness of an inliner
\end{itemize}

}

%===============================================================================

\frame{
\frametitle{Plotkin, Scott, and Engeler's Models}

\begin{description}
\item[Plotkin 1972:] $\mathcal{P}(\mathbb{V})$ where
\[
  \mathbb{V} = \mathbb{Z} + \FSET{\mathbb{V}} \times \FSET{\mathbb{V}}
\]
\item[Scott 1976:] $\mathcal{P}(\mathbb{N})$
%% $\mathcal{P}(\mathbb{N})$ where
%%   $\mathbb{N}$ encodes Engeler's version of $\mathbb{V}$
\item[Engeler 1981:] $\mathcal{P}(\mathbb{V})$ where
\[
  \mathbb{V} = \mathbb{Z} + \FSET{\mathbb{V}} \times \mathbb{V}
\]

\item[Siek 2017:] $\mathcal{P}(\mathbb{V})$ where
\[
 \mathbb{V} = \mathbb{Z} + \mathcal{P}_{f}(\mathbb{V} \times \mathbb{V}) 
\]
\end{description}
}
%===============================================================================

%% \frame{
%% \frametitle{Plotkin's Model}


%% \fbox{$\SEM{e}\rho \in \SET{\mathbb{V}}$}
%% \begin{align*}
%% \SEM{ n }\rho &= \{ n \} \\
%% \SEM{ e_1 \oplus e_2 }\rho &= \{  n_1 \oplus n_2 \mid 
%%    n_1 \in \SEM{ e_1 }\rho \land n_2 \in \SEM{ e_2 }\rho \} \\
%% \SEM{ x }\rho &= \rho(x) \\
%% \SEM{ \lam{x} e }\rho &= 
%%   \{ (V,V') \mid  V' \subseteq \SEM{ e }\rho(x{:=}V) \} \\
%% \SEM{ e_1\;e_2 }\rho &= \bigcup \left\{ V' \, \middle| 
%%    \begin{array}{l}
%%    \exists V.\, (V,V') {\in} \SEM{ e_1 }\rho \land V {\subseteq} \SEM{ e_2 }\rho 
%%    \end{array}
%% \right\} \\
%% \SEM{\IF e_1 \THEN e_2 \ELSE e_3}\rho &=
%% \left\{ v\, \middle|\,
%%   \begin{array}{l}
%%    \exists n.\, n \in \SEM{e_1}\rho \\
%%    \land\, (n\neq 0 \implies v \in \SEM{e_2}\rho)\\
%%    \land\, (n=0 \implies v \in \SEM{e_3}\rho)
%%   \end{array}
%%   \right\}
%% \end{align*}

%% }

%===============================================================================

\frame{
\frametitle{Scott and Engeler's Model}

\[
  \mathbb{V} = \mathbb{Z} + \FSET{\mathbb{V}} \times \mathbb{V}
\]

\fbox{$\SEM{-} : \mathbb{E} \to (\mathbb{X} \to \rd{\FSET{\mathbb{V}}}) \to  \SET{\mathbb{V}}$}
\begin{align*}
\SEM{ n }\rho &= \{ n \} \\
\SEM{ e_1 \oplus e_2 }\rho &= \{  n_1 \oplus n_2 \mid 
   n_1 \in \SEM{ e_1 }\rho \text{ and } n_2 \in \SEM{ e_2 }\rho \} \\
\SEM{ x }\rho &= \rho(x) \\
\SEM{ \lam{x} e }\rho &= 
  \{ \rd{(V,v')} \mid  v' \in \SEM{ e }\rho(x{:=}\rd{V}) \} \\
\SEM{ e_1\;e_2 }\rho &= \left\{ v' \, \middle| 
   \begin{array}{l}
   \exists V.\, \rd{(V,v')} \in \SEM{ e_1 }\rho \text{ and } \rd{V \subseteq\;} \SEM{ e_2 }\rho
   \end{array}
\right\} \\
\SEM{\IF e_1 \THEN e_2 \ELSE e_3 \FI}\rho &=
\left\{ v\, \middle|\,
  \begin{array}{l}
   \exists n,\, n \in \SEM{e_1}\rho, \\
   (n= 0 \implies v \in \SEM{e_2}\rho), \text{ and}\\
   (n\neq 0 \implies v \in \SEM{e_3}\rho)
  \end{array}
  \right\}
\end{align*}

}

%===============================================================================

\frame{
\frametitle{Outline}

\begin{itemize}
\item A simple semantics of $\lambda$-calculus in two
  forms:\\ relational and denotational.
\item Plotkin, Scott, and Engeler's models
\item \textbf{Correctness of an inliner}
\end{itemize}
}

%===============================================================================
\frame{
\frametitle{A Function Inliner}

\fbox{$\mathcal{O}(e,k) = e'$}
\begin{align*}
  \mathcal{O}(x, k) &= x \\
  \mathcal{O}(n, k) &= n \\
  \mathcal{O}(e_1 \oplus e_2, k) &=
  \begin{cases}
    n_1 \SEM{\oplus} n_2 & \text{if }
    \mathcal{O}(e_1,k) {=} n_1, 
    \mathcal{O}(e_2,k) {=} n_2 \\
    \mathcal{O}(e_1, k) \oplus \mathcal{O}(e_2,k)
    & \text{otherwise}
  \end{cases} \\
  \mathcal{O}(\lambda x.\, e, k) &=
    \lambda x.\, \mathcal{O}(e, k) \\
  \mathcal{O}(e_1\app e_2, k) &=
  \begin{cases}
    \mathcal{O}(e[v_2/x], k{-}1)
    & \text{if } k \geq 1, \mathcal{O}(e_1, k) = \lambda x.\, e \\
     & \text{and } \mathcal{O}(e_2,k) = v_2 \\
    \mathcal{O}(e_1, k) \app \mathcal{O}(e_2, k) & \text{otherwise}
  \end{cases} 
\end{align*}

}

%===============================================================================

\frame{
\frametitle{Correctness of the Inliner}

\begin{lemma}[Optimizer Preserves Denotations]
  \label{lem:O2-correct-aux}
  $\SEM{\mathcal{O}(e,k)} = \SEM{e}$
\end{lemma}
\begin{proof} By induction on $\mathcal{O}$.
\begin{itemize}
\item Case $x$: $\SEM{\mathcal{O}(x,k)} = \SEM{x}$  by def. $\mathcal{O}$.
\item Case $n$: $\SEM{\mathcal{O}(n,k)} = \SEM{n}$  by def. $\mathcal{O}$.
\item Case $e_1 \oplus e_2$: \\
  Suppose $\mathcal{O}(e_1,k) = n_1, \mathcal{O}(e_2,k) = n_2$.\\
  So $\mathcal{O}(e_1 \oplus e_2) = n_1 \SEM{\oplus} n_2$.\\
%  NTS $\SEM{n_1 \SEM{\oplus} n_2} = \SEM{e_1 \oplus e_2}$.\\
%  By IH, $\SEM{e_1} = \SEM{n_1}$, $\SEM{e_2} = \SEM{n_2}$.\\
  $\SEM{n_1 \SEM{\oplus} n_2} =
   \SEM{n_1 \oplus n_2} =
   \SEM{e_1 \oplus e_2}$ by IH and congruence.
   Suppose not.
   So $\mathcal{O}(e_1 \oplus e_2) = \mathcal{O}(e_1,k) \oplus \mathcal{O}(e_2,k)$.\\
   $\SEM{\mathcal{O}(e_1,k) \oplus \mathcal{O}(e_2,k)} = 
      \SEM{e_1 \oplus e_2}$ by IH and congruence.

\item Cont'd on next slide.
      
\end{itemize}
\end{proof}

}

%===============================================================================

\frame{
  \frametitle{Correctness of the Inliner, cont'd}

\begin{itemize}
\item Case $\lam{x}e$:\\
  So $\mathcal{O}(\lam{x}e,k) = \lam{x} \mathcal{O}(e,k)$ \\
  By the IH $\SEM{\mathcal{O}(e,k)} = \SEM{e}$. \\
  Therefore $\SEM{\lam{x} \mathcal{O}(e,k)} = \SEM{\lam{x}e}$
  by congruence.

\item Case $e_1 \app e_2$: \\
  Suppose $\mathcal{O}(e_1,k) = \lam{x} e, \mathcal{O}(e_2,k)=v_2$. \\
  So $\mathcal{O}(e_1 \app e_2,k) = \mathcal{O}(e[v_2/x],k-1)$.
  \begin{align*}
  \SEM{\mathcal{O}(e[v_2/x],k-1)}
  &= \SEM{e[v_2/x]} & \text{(by IH)}\\
  &= \SEM{(\lam{x}e) \app v_2} & \text{($\beta$ preserves meaning)}\\
  &= \SEM{e_1 \app e_2} & \text{(by IH and congruence)}
  \end{align*}
  Suppose not.
  So $\mathcal{O}(e_1 \app e_2,k) = \mathcal{O}(e_1,k) \app \mathcal{O}(e_2,k)$.
  We conclude that
  $\SEM{\mathcal{O}(e_1,k) \app \mathcal{O}(e_2,k)}
  = \SEM{e_1 \app e_2}$
  by IH and congruence.
  
\end{itemize}

}

%===============================================================================

\frame{
  \frametitle{Recap \& Further Reading}

  \begin{itemize}
  \item $\lambda$ can be modeled with finite graphs! 
  \item Ingredients: subsumption + $\mathcal{P}$.
  \item This insight goes back to Plotkin, Scott, and Engeler.
  \item Several views of the semantics: relational, denotational,
    non-deterministic interpreter$^\dagger$, intersection
    types$^\dagger$
  \item Correctness of an inliner.
  \item Semantics for System F and semantic type soundness$^\dagger$
  \item The simple semantics is deterministic (see my blog)
  \end{itemize}

  $^\dagger$ not discussed but in the draft on arXiv:\\
  Revisiting Elementary Denotational Semantics
     \url{https://arxiv.org/abs/1707.03762}.\\
  Isabelle proofs available in the Archive of Formal Proof.

}

%===============================================================================

\frame{
\frametitle{Burning Questions}

\begin{itemize}
\item Why didn't graph models become popular?
\item Are there any killer reasons to not use graph models?
\item Can graph models scale to handle real languages?
\item Do graph models make proofs easier?
  \begin{itemize}
  \item compiler correctness
  \item program equivalence
  \end{itemize}
\item Can we distinguish errors from non-termination 
   like we can in reduction semantics? 
\end{itemize}

}


\end{document}

% LocalWords:  titlepage containsverbatim frametitle  Siek Taha lstlisting
% LocalWords:  Rightarrow IU CBV Denotational SECD Landin Plotkin et
% LocalWords:  texttt Rightarrow texttt texttt Rightarrow Findler deriv Wadler
% LocalWords:  Longrightarrow Longrightarrow circ Wrigstad ldots inc bytecode
% LocalWords:  includegraphics invokedynamic switchpoints unboxed xymatrix rrr
% LocalWords:  vspace newsavebox DistExample lrbox linewidth mathtt mathtt emph
% LocalWords:  vdash usebox mathbf mathit longmapsto mathsf subtyping emptyset
% LocalWords:  Henglein's footnotesize Drossopoulou Igarashi Gronski Dimoulas
% LocalWords:  varphi eval Jython microbenchmarks Fibonnaci vitousek bharadwaj
% LocalWords:  Shashank  Barendregt al Coppo Dezani Salle Damas Rees
%%  LocalWords:  Felleisen Compositional denotational rcl de Atkey AO
%%  LocalWords:  lcl lccl Heh subsumption wrt Abramsky's btw BCDC BCD
%%  LocalWords:  CDCHL Abramsky BDS iff Inhabitation CDHL HL Bool FV
%%  LocalWords:  darkgreen Egidi Honsell Ronchi Rocca preorder Alessi
%%  LocalWords:  Barbanera Ciancaglini nts dom Sem approximants ctx
%%  LocalWords:  Hyland CDS polymorphism
