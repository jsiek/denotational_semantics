\documentclass{article}
\usepackage{amsthm}
\usepackage{amsmath}
\usepackage{amssymb}
\usepackage{stmaryrd}
\usepackage{hyperref}

\newcommand{\lam}[1]{\lambda #1.\,}
\newcommand{\app}[0]{\;}
\newcommand{\IF}[0]{\textbf{if}\;}
\newcommand{\THEN}[0]{\;\textbf{then}\;}
\newcommand{\ELSE}[0]{\;\textbf{else}\;}
\newcommand{\ESEM}[1]{\mathcal{E}[\![ #1 ]\!]}
\newcommand{\FSEM}[1]{\mathcal{F}[\![ #1 ]\!]}
\newcommand{\SET}[1]{\mathcal{P}(#1)}


\title{Intersection Types, Cut Elimination, and the Functional
  Character of the Lambda Calculus}

\author{Jeremy G. Siek}

\begin{document}
\maketitle

Last December I proved that my graph model of the lambda calculus,
once suitable restricted, is deterministic. That is, I defined a
notion of \emph{consistency} between values, written $v_1 \sim v_2$,
and showed that any two outputs of the same program are consistent. \\

\noindent \textbf{Theorem} (Determinism of $\mathcal{E}$)
  If $v \in \ESEM{e}\rho$, $v' \in \ESEM{e}\rho'$, and $\rho \sim
  \rho'$, then $v \sim v'$. \\

\noindent Recall that values are integers or finite relations;
consistency for integers is equality and consistency for relations
means mapping consistent inputs to consistent outputs.

Having proved the Determinism Theorem, I thought it would be
straightforward to prove the following related theorem about the join
of two values. \\

\noindent \textbf{Theorem} (Join in $\mathcal{E}$)
  If $v \in \ESEM{e}\rho$, $v' \in \ESEM{e}\rho'$, and $\rho \sim
  \rho'$, \\
  then $v \sqcup v' \in \ESEM{e}(\rho\sqcup\rho')$. \\

Recall that we have defined a partial order $\sqsubseteq$ on values,
and that, in most partial orders, there is a close connection between
notions of consistency and least upper bounds (joins). One typically
has that $v \sim v'$ iff $v \sqcup v'$ exists. So my thinking was that
it should be easy to adapt my proof of the determinism theorem to
prove this theorem about joins, and I set out hoping to finish in a
couple weeks. Hah! Here we are 8 months later and the proof is
complete, but my journey to find the proof was rather circuitous and
ended up depending on a recent result about intersection types and cut
elimination from Olivier Laurent. In this blog post I'll try to
recount the journey and describe the proof, hopefully remembering all
the challenges and motivations that led to the proof.

Many of the challenges revolved around the definitions of
$\sqsubseteq$, consistency, and $\sqcup$. Given that I already had
definitions for $\sqsubseteq$ and consistency, the obvious thing to
try was to define $\sqcup$ such that it would be the least upper bound
of $\sqsubseteq$. So I arrived at this partial function:
\begin{align*}
  n \sqcup n &= n \\
  f_1 \sqcup f_2 &= f_1 \cup f_2
\end{align*}
Now suppose we prove the Join Theorem by induction on $e$ and consider
the case for application: $e = (e_1 \app e_2)$.
We have
\begin{itemize}
\item $f \in \ESEM{e_1}\rho$, $v_2 \in \ESEM{e_2}\rho$,  
  $v_3 \mapsto v_4 \in f$, $v_3 \sqsubseteq v_2$, and $v \sqsubseteq v_4$
  for some $f, v_2, v_3, v_4$.
\item $f' \in \ESEM{e_2}\rho'$, $v'_2 \in \ESEM{e_2}\rho'$,   
  $v'_3 \mapsto v'_4 \in f$, $v'_3 \sqsubseteq v'_2$, and $v' \sqsubseteq v'_4$
  for some $f', v'_2, v'_3, v'_4$.
\end{itemize}
By the induction hypothesis we have $f \sqcup f' \in \ESEM{e_1}$
and $v_2 \sqcup v'_2 \in \ESEM{e_2}$.
We need to show that 
\[
   v''_3 \mapsto v''_4 \in f \sqcup f' 
   \qquad
   v''_3 \sqsubseteq v_2 \sqcup v'_2
   \qquad
   v \sqcup v' \sqsubseteq v''_4
\]
But we have a problem, given our definition of $\sqcup$ in terms of
set union, there won't necissarily be a single entry in $f \sqcup f'$
that combines the information from both $v_3 \mapsto v_4$ and $v'_3
\mapsto v'_4$. After all, $f \sqcup f'$ contains all the entries of
$f$ and all the entries of $f'$, but no entries that mix together the
information from entries in $f$ and $f'$.

At this point I started thinking that my definitions of $\sqsubseteq$,
consistency, and $\sqcup$ were too simple, and that I needed to
incorporate ideas from the literature on filter models and
intersection types. As I've written about previously, my graph model
corresponds to a particular intersection type system, and perhaps a
different intersection type system would do the job. Recall that the
correspondence goes as follows: values correspond to types,
$\sqsubseteq$ corresponds to subtyping $<:$ (in reverse), and $\sqcup$
corresponds to intersection $\sqcap$. The various intersection type
systems primarily differ in their definitions of subtyping.  Given the
above proof attempt, I figured that I would need usual
co/contra-variant rule for function types and also the following rule
for distributing intersections over function types.
\[
  (A\to B) \sqcap (A \to C) <: A \to (B \sqcap C)
\]
This distributivity rule enables the ``mixing'' of information from
two different entries.

So I defined types as follows:
\[
  A,B,C,D ::= n \mid A \to B \mid A \sqcap B
\]
and defined subtyping to follow that of the BCD intersection type
system.
\begin{gather*}
A <: A \qquad \frac{A <: B \quad B <: C}{A <: C} \\[2ex]
A \sqcap B <: A \qquad
A \sqcap B <: B \qquad
\frac{C <: A \quad C <: B}{C <: A \sqcap B} \\[2ex]
\frac{C <: A \quad B <: D}{A \to B <: C \to D}
\qquad
(A\to B) \sqcap (A \to C) <: A \to (B \sqcap C)
\end{gather*}
I then adapted the definition of consistency to work over types.
Because of the use of negation in this definition, it is easier to
define consistency as a recursive function in Isabelle instead of as
an inductively defined relation.
\begin{align*}
 n \sim n' &= (n = n') \\
 n \sim (C \to D) &= \mathit{false} \\
 n \sim (C \sqcap D) &= n \sim C \text{ and } n \sim D \\
 (A \to B) \sim n' &= \mathit{false} \\
 (A \to B) \sim (C \to D) &= 
    (A \sim C \text{ and } B \sim D) \text{ or } A \not\sim C \\
 (A \to B) \sim (C \sqcap D) &=
    (A \to B) \sim C \text{ and } (A \to B) \sim D \\
 (A \sqcap B) \sim n' &= A \sim n' \text{ and } B \sim n' \\
 (A \sqcap B) \sim (C \sqcap D) &= 
    A \sim C \text{ and } A \sim D \text{ and } 
    B \sim C \text{ and } B \sim D
\end{align*}

Turning back to the Join Theorem, let us restate it in terms of an
intersection type system. Instead of using the letter $\rho$ for
environments, we shall switch to $\Gamma$ because they now contain
types instead of values. \\

\noindent \textbf{Theorem} (Intersection Introduction is Admissible)
  If $\Gamma \vdash e : A$, $\Gamma' \vdash e : B$, and $\Gamma \sim
  \Gamma'$, 
  then $\Gamma\sqcap\Gamma' \vdash e : A \sqcap B$. \\

By restating the theorem in terms of intersection types, we have
essentially arrived at the rule for intersection introduction.  In
other words, if we can prove this theorem we will have shown that the
intersection introduction rule is admissible in our system.

While the change to intersection types enabled this top-level proof to
go through, I got stuck on one of the lemmas that it requires, which
is an adaptation of Proposition 3 of the prior blog post. \\

\noindent \textbf{Lemma} (Consistency and Subtyping)
\begin{enumerate}
\item  If $A \sim B$, $A <: C$, and $B <: D$,
  then $C \sim D$.
\item If $A \not\sim B$, $C <: A$, $D <: B$, then $C \not\sim D$.
\end{enumerate}
In particular, I got stuck in the cases where the subtyping $A <: C$
or $B <: D$ was derived using the transitivity subtyping rule.



\end{document}

%%  LocalWords:  denotationally ctx subexpression eq denot lrcl env
