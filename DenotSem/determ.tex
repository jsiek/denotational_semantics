\documentclass{article}
\usepackage{amsthm}
\usepackage{amsmath}
\usepackage{amssymb}
\usepackage{stmaryrd}
\usepackage{hyperref}

\newcommand{\lam}[1]{\lambda #1.\,}
\newcommand{\app}[0]{\;}
\newcommand{\IF}[0]{\mathrm{if}\;}
\newcommand{\THEN}[0]{\;\mathrm{then}\;}
\newcommand{\ELSE}[0]{\;\mathrm{else}\;}
\newcommand{\SEM}[1]{[\![ #1 ]\!]}
\newcommand{\ESEM}[1]{\mathcal{E}\SEM{#1}}
\newcommand{\FSEM}[1]{\mathcal{F}\SEM{#1}}
\newcommand{\SET}[1]{\mathcal{P}(#1)}
\newcommand{\WF}[1]{\mathsf{wf}(#1)}
\newcommand{\UP}[1]{\mathord{\uparrow} #1}
\newcommand{\ATOMS}[1]{\mathit{atoms}(#1)}
\newcommand{\sqinter}[0]{\bigsqcap}


\title{Intersection Types, Sub-formula Property, and the Functional
  Character of the Lambda Calculus}

\author{Jeremy G. Siek}

\begin{document}
\maketitle

Last December I proved that my graph model of the lambda calculus,
once suitable restricted, is deterministic. That is, I defined a
notion of \emph{consistency} between values, written $v_1 \sim v_2$,
and showed that any two outputs of the same program are consistent. \\

\noindent \textbf{Theorem} (Determinism)\\
  If $v \in \ESEM{e}\rho$, $v' \in \ESEM{e}\rho'$, and $\rho \sim
  \rho'$, then $v \sim v'$. \\

\noindent Recall that values are integers or finite relations;
consistency for integers is equality and consistency for relations
means mapping consistent inputs to consistent outputs. I then
restricted values to be well formed, meaning that they must be
consistent with themselves (and similarly for their parts).


Having proved the Determinism Theorem, I thought it would be
straightforward to prove the following related theorem about the join
of two values. \\

\noindent \textbf{Theorem} (Join)\\
If $v \in \ESEM{e}\rho$, $v' \in \ESEM{e}\rho'$,
$\rho$ is well formed, $\rho'$ is well formed,
and $\rho \sim \rho'$, \\
then $v \sqcup v' \in \ESEM{e}(\rho\sqcup\rho')$. \\

\noindent I am particularly interested in this theorem because
$\beta$-equality can be obtained as a corollary.
\[
\ESEM{(\lam{x}e)\app e'}\rho = \ESEM{[x{:=}e']e}\rho
\]
This would enable the modeling of the call-by-name $\lambda$-calculus
and it would also enable the use of $\beta$-equality in a
call-by-value setting when $e'$ is terminating (instead of restricting
$e'$ to be a syntactic value).


Recall that we have defined a partial order $\sqsubseteq$ on values,
and that, in most partial orders, there is a close connection between
notions of consistency and least upper bounds (joins). One typically
has that $v \sim v'$ iff $v \sqcup v'$ exists. So my thinking was that
it should be easy to adapt my proof of the Determinism Theorem to
prove the Join Theorem, and I set out hoping to finish in a couple
weeks. Hah! Here we are 8 months later and the proof is complete; it
was a long journey that ended up depending on a result that was
published just this summer, concerning intersection types, the
sub-formula property, and cut elimination by Olivier Laurent. In this
blog post I'll try to recount the journey and describe the proof,
hopefully remembering the challenges and motivations.
Here is a tar ball of the mechanization in Isabelle and in pdf form.

Many of the challenges revolved around the definitions of
$\sqsubseteq$, consistency, and $\sqcup$. Given that I already had
definitions for $\sqsubseteq$ and consistency, the obvious thing to
try was to define $\sqcup$ such that it would be the least upper bound
of $\sqsubseteq$. So I arrived at this partial function:
\begin{align*}
  n \sqcup n &= n \\
  f_1 \sqcup f_2 &= f_1 \cup f_2
\end{align*}
Now suppose we prove the Join Theorem by induction on $e$ and consider
the case for application: $e = (e_1 \app e_2)$. From $v \in \ESEM{e_1
  \app e_2}$ and $v' \in \ESEM{e_1 \app e_2}$ we have
\begin{itemize}
\item $f \in \ESEM{e_1}\rho$, $v_2 \in \ESEM{e_2}\rho$,  
  $v_3 \mapsto v_4 \in f$, $v_3 \sqsubseteq v_2$, and $v \sqsubseteq v_4$
  for some $f, v_2, v_3, v_4$.
\item $f' \in \ESEM{e_2}\rho'$, $v'_2 \in \ESEM{e_2}\rho'$,   
  $v'_3 \mapsto v'_4 \in f$, $v'_3 \sqsubseteq v'_2$, and $v' \sqsubseteq v'_4$
  for some $f', v'_2, v'_3, v'_4$.
\end{itemize}
By the induction hypothesis we have $f \sqcup f' \in \ESEM{e_1}$
and $v_2 \sqcup v'_2 \in \ESEM{e_2}$.
We need to show that 
\[
   v''_3 \mapsto v''_4 \in f \sqcup f' 
   \qquad
   v''_3 \sqsubseteq v_2 \sqcup v'_2
   \qquad
   v \sqcup v' \sqsubseteq v''_4
\]
But here we have a problem. Given our definition of $\sqcup$ in terms
of set union, there won't necessarily be a single entry in $f \sqcup
f'$ that combines the information from both $v_3 \mapsto v_4$ and
$v'_3 \mapsto v'_4$. After all, $f \sqcup f'$ contains all the entries
of $f$ and all the entries of $f'$, but the set union operation does
not mix together information from entries in $f$ and $f'$ to form new
entries.

\section{Intersection Types to the Rescue}

At this point I started thinking that my definitions of $\sqsubseteq$,
consistency, and $\sqcup$ were too simple, and that I needed to
incorporate ideas from the literature on filter models and
intersection types. As I've written about previously, my graph model
corresponds to a particular intersection type system, and perhaps a
different intersection type system would do the job. Recall that the
correspondence goes as follows: values correspond to types,
$\sqsubseteq$ corresponds to subtyping $<:$ (in reverse), and $\sqcup$
corresponds to intersection $\sqcap$. The various intersection type
systems primarily differ in their definitions of subtyping.  Given the
above proof attempt, I figured that I would need the usual
co/contra-variant rule for function types and also the following rule
for distributing intersections over function types.
\[
  (A\to B) \sqcap (A \to C) <: A \to (B \sqcap C)
\]
This distributivity rule enables the ``mixing'' of information from
two different entries.

So I defined types as follows:
\[
  A,B,C,D ::= n \mid A \to B \mid A \sqcap B
\]
and defined subtyping according to the BCD intersection type system
(\emph{Lambda Calculus with Types}, Barendregt et al. 2013).
\begin{gather*}
A <: A \qquad \frac{A <: B \quad B <: C}{A <: C} \\[2ex]
A \sqcap B <: A \qquad
A \sqcap B <: B \qquad
\frac{C <: A \quad C <: B}{C <: A \sqcap B} \\[2ex]
\frac{C <: A \quad B <: D}{A \to B <: C \to D}
\qquad
(A\to B) \sqcap (A \to C) <: A \to (B \sqcap C)
\end{gather*}
I then adapted the definition of consistency to work over types.
(Because this definition uses negation, it is easier to define
consistency as a recursive function in Isabelle instead of as an
inductively defined relation.)
\begin{align*}
 n \sim n' &= (n = n') \\
 n \sim (C \to D) &= \mathit{false} \\
 n \sim (C \sqcap D) &= n \sim C \text{ and } n \sim D \\
 (A \to B) \sim n' &= \mathit{false} \\
 (A \to B) \sim (C \to D) &= 
    (A \sim C \text{ and } B \sim D) \text{ or } A \not\sim C \\
 (A \to B) \sim (C \sqcap D) &=
    (A \to B) \sim C \text{ and } (A \to B) \sim D \\
 (A \sqcap B) \sim n' &= A \sim n' \text{ and } B \sim n' \\
 (A \sqcap B) \sim (C \sqcap D) &= 
    A \sim C \text{ and } A \sim D \text{ and } 
    B \sim C \text{ and } B \sim D
\end{align*}

Turning back to the Join Theorem, I restated it in terms of the
intersection type system and re-branded it the Meet Theorem. Instead of
using the letter $\rho$ for environments, we shall switch to $\Gamma$
because they now contain types instead of values. \\

\noindent \textbf{Theorem} (Meet)\\
  If $\Gamma \vdash e : A$, $\Gamma' \vdash e : B$, and $\Gamma \sim
  \Gamma'$, 
  then $\Gamma\sqcap\Gamma' \vdash e : A \sqcap B$. \\

By restating the theorem in terms of intersection types, we have
essentially arrived at the rule for intersection introduction.  In
other words, if we can prove this theorem we will have shown that the
intersection introduction rule is admissible in our system.

While the switch to intersection types and subtyping enabled this
top-level proof to go through, I got stuck on one of the lemmas that
it requires, which is an adaptation of Proposition 3 of the prior blog
post. \\

\noindent \textbf{Lemma} (Consistency and Subtyping)
\begin{enumerate}
\item  If $A \sim B$, $A <: C$, and $B <: D$,
  then $C \sim D$.
\item If $A \not\sim B$, $C <: A$, $D <: B$, then $C \not\sim D$.
\end{enumerate}
In particular, I got stuck in the cases where the subtyping $A <: C$
or $B <: D$ was derived using the transitivity rule.

\section{Subtyping and the Sub-formula Property}

For a long time I've disliked definitions of subtyping in which
transitivity is given as a rule instead of proved as a theorem.  There
are several reasons for this: a subtyping algorithm can't directly
implement a transitivity rule (or any rule that is not syntax
directed), reasoning by induction or cases (inversion) is more
difficult, and it is redundant. Furthermore, the presence of the
transitivity rule means that subtyping does not satisfy the sub-formula
property. This term sub-formula property comes from logic, and means
that a derivation (proof) of a formula only mentions propositions that
are a part of the formulate to be proved. The transitivity rule breaks
this property because the type $B$ comes out of nowhere, it is not
part of $A$ or $C$, the types in the conclusion of the rule.

So I removed the transitivity rule and tried to prove transitivity.
For most type systems, proving the transitivity of subtyping is
straightforward. But I soon realized that the addition of the
distributivity rule makes it significantly more difficult.  After
trying and failing to prove transitivity for some time, I resorted to
reading the literature. Unfortunately, it turns out that none of the
published intersection type systems satisfied the sub-formula property
and vast majority of them included the transitivity rule.  However,
there was one paper that offered some hope.  In a 2012 article in
Fundamenta Informaticae titled \emph{Intersection Types with Subtyping
  by Means of Cut Elimination}, Olivier Laurent defined subtyping
without transitivity and instead proved it, but his system still did
not satisfy the sub-formula property because of an additional rule for
function types. Nevertheless, Olivier indicated that he was interested
in finding a version of the system that did, writing
\begin{quote}
``it would be much nicer and much more natural to go through a
  sub-formula property''
\end{quote}

A lot of progress can happen in six years, so I sent an email to
Olivier. He replied,
\begin{quote}
``Indeed! I now have two different sequent-calculus systems which are
  equivalent to BCD subtyping and satisfy the sub-formula property.  I
  am currently writing a paper on this but it is not ready yet.''
\end{quote}
and he attached the paper draft and the Coq mechanization. What great
timing!  Furthermore, Olivier would be presenting the paper, titled
\emph{Intersection Subtyping with Constructors}, at the Workshop on
Intersection Types and Related System in Oxford on July 8, part of the
Federated Logic Conference (FLOC). I was planning to attend FLOC
anyways, for the DOMAINS workshop to celebrate Dana Scott's 85th
birthday.

Olivier's systems makes two important changes compared to prior work:
he combines the distributivity rule and the usual arrow rule into a
single elegant rule, and to enable this, he generalizes the form of
subtyping from $A <: B$ to $A_1,\ldots,A_n \vdash B$, which should be
interpreted as meaning $A_1 \sqcap \cdots \sqcap A_n <: B$.  Having a
sequence of formulas (types) on the left is characteristic of proof
systems in logic, including both natural deduction systems and
sequence calculi. (Sequent calculi, in addition, typically have a
sequence on the right that means the disjunction of the formulas.)
Here is one of Olivier's systems, adapted to my setting, which I'll
describe below. Let $\Gamma$ range over sequences of types.
\begin{gather*}
  \frac{\Gamma_1, \Gamma_2 \vdash A}
       {\Gamma_1 , n, \Gamma_2 \vdash A} \qquad
  \frac{\Gamma_1, \Gamma_2 \vdash A}
       {\Gamma_1 , B \to C, \Gamma_2 \vdash A}
   \\[2ex]
   \frac{\Gamma \vdash A \quad \Gamma \vdash B}{\Gamma \vdash A \sqcap B}
   \qquad
   \frac{\Gamma_1,B,C,\Gamma_2 \vdash A}{\Gamma_1,B\sqcap C,\Gamma_2 \vdash A}
   \\[2ex]
   \frac{}{n \vdash n}
   \qquad
   \frac{A \vdash C_1, \ldots, C_n \quad
         D_1, \ldots, D_n \vdash B}
        {C_1\to D_1,\ldots, C_n\to D_n \vdash A \to B}
\end{gather*}
The first two rules are weakening rules for singleton integers and
function types. There is no weakening rule for intersections.  The
third and fourth rules are introduction and elimination rules for
intersection. The fifth rule is reflexivity for integers, and the last
is the combined rule for function types.

The combined rule for function types says that the intersection of a
sequence of function types $\sqinter_{i=1\ldots n} (C_i\to D_i)$ is a
subtype of $A \to B$ if
\[
  A <: \sqinter_{i\in\{1\ldots n\}} C_i
  \qquad \text{and}\qquad
  \sqinter_{i\in\{1\ldots n\}} D_i <: B
\]
Interestingly, the inversion principle for this rule is the
$\beta$-sound property described in Chapter 14 of \emph{Lambda
  Calculus with Types} by Barendregt et al., and is the key to proving
$\beta$-equality. In Olivier's system, $\beta$-soundness falls out
immediately, instead of by a somewhat involved proof.

The regular subtyping rule for function types is simply an instance of
the combined rule in which the sequence on the left contains just one
function type.

The next step for me was to enter Olivier's definitions into Isabelle
and prove transitivity via cut elimination. That is, I needed to prove
the following generalized statement via a sequence of lemmas laid out
by Olivier in his draft.\\

\noindent \textbf{Theorem} (Cut Elimination)\\
  If $\Gamma_2 \vdash B$ and $\Gamma_1,B,\Gamma_3 \vdash C$,
  then $\Gamma_1,\Gamma_2,\Gamma_3 \vdash C$. \\

The transitivity rule is the instance of cut elimination where
$\Gamma_2 = A$ and both $\Gamma_1$ and $\Gamma_3$ are empty.

Unfortunately, I couldn't resist making changes to Olivier's subtyping
system as I entered it into Isabelle, which cost me considerable time.
Some of Olivier's lemmas show that the collection of types on the
left, that is, the $A's$ in $A_1,\ldots, A_n \vdash B$, behave like a
set instead of a sequence. I figured that if the left-hand-side was
represented as a set, then I would be able to bypass several lemmas
and obtain a shorter proof. I got stuck in proving Lemma $\cap L_e$
which states that $\Gamma_1,A\sqcap B,\Gamma_2 \vdash C$ implies
$\Gamma_1,A, B,\Gamma_2 \vdash C$. Olivier's subtyping rules are
carefully designed to minimize the amount of overlap between the
rules, and switching to a set representation increases the amount of
overlap, making the proof of this lemma more difficult (perhaps
impossible?).

So after struggling with the set representation for some time, I went
back to sequences and was able to complete the proof of cut
elimination, with a little help from Olivier at FLOC.  I proved the
required lemmas in the following order. \\

\noindent \textbf{Lemma} (Weakening)\\
  If $\Gamma_1,\Gamma_2 \vdash A$,
  then $\Gamma_1,B,\Gamma_2 \vdash A$. \\
  (Proved by induction on $A$.) \\

\noindent \textbf{Lemma} (Axiom)\\
$A \vdash A$ \\
  (Proved by induction on $A$.)\\

\noindent \textbf{Lemma} (Permutation)\\
  If $\Gamma_1 \vdash A$ and $\Gamma_2$ is a permutation of $\Gamma_1$,
  then $\Gamma_2 \vdash A$. \\
  (Proved by induction on the derivation of $\Gamma_1 \vdash A$,
   using many lemmas about permutations.)\\
  
\noindent \textbf{Lemma} ($\cap L_e$) \\
If $\Gamma_1,A\sqcap B,\Gamma_2 \vdash C$, then
$\Gamma_1,A, B,\Gamma_2 \vdash C$. \\
(Proved by induction on the derivation of $\Gamma_1,A\sqcap B,\Gamma_2
\vdash C$.)\\

\noindent \textbf{Lemma} (Collapse Duplicates) \\
%
If $\Gamma_1,A,A,\Gamma_2 \vdash C$, then $\Gamma_1,A,\Gamma_2 \vdash
C$. \\
%
(This is proved by well-founded induction on the lexicographical
ordering of the pair $(n,k)$ where $n$ is the size of $A$ and $k$ is
the depth of the derivation of $\Gamma_1,A,A,\Gamma_2 \vdash C$. Proof
assistants such as Isabelle and Coq do not directly provide the depth
of a derivation, but the depth can be manually encoded as an extra
argument of the relation, as in $\Gamma_1,A,A,\Gamma_2 \vdash_k C$.) \\

The Cut Elimination Theorem is then proved by well-founded induction
on the triple $(n,k_1,k_2)$ where $n$ is the size of B, $k_1$ is the
depth of the derivation of $\Gamma_2 \vdash B$, and $k_2$ is the depth
of the derivation of $\Gamma_1,B,\Gamma_3 \vdash C$.

We define subtyping as follows.
\[
  A <: B \quad = \quad A \vdash B
\]

The BCD subtyping rules and other derived rules follow from the above
lemmas.\\

\noindent \textbf{Proposition} (Properties of Subtyping) 
\begin{enumerate}
\item $A <: A$.
\item If $A <: B$ and $B <: C$, then $A <: C$. 
\item If $C <: A$ and $B <: D$, then $A \to B <: C \to D$.
\item If $A_1 <: B$, then $A_1 \sqcap A_2 <: B$.
\item If $A_2 <: B$, then $A_1 \sqcap A_2 <: B$.
\item If $B <: A_1$ and $B <: A_2$, then $B <: A_1 \sqcap A_2$.
\item If $A <: C$ and $B <: D$, then $A \sqcap B <: C \sqcap D$.
\item $(A\to B) \sqcap (A \to C) <: A \to (B \sqcap C)$.
\item $(A \to C) \sqcap (B \to D) <: (A\sqcap B) \to (C \sqcap D)$
\end{enumerate}
  

\section{Consistency and Subtyping, Resolved}

Recall that my switch to intersection types was motivated by my
failure to prove the Consistency and Subtyping Lemma.  We now return
to the proof of that Lemma. We start with a handful of lemmas that are
needed for that proof. \\

\noindent \textbf{Lemma} (Consistency is Symmetric and Reflexive) 
%
\begin{enumerate}
\item If $A \sim B$, then $B \sim A$.
\item If $\WF{A}$, then $A \sim A$.
\end{enumerate}

It will often be convenient to decompose a type into its set of
\emph{atoms}, defined as follows.
\begin{align*}
  \ATOMS{n} &= \{ n \} \\
  \ATOMS{A\to B} &= \{ A \to B \} \\
  \ATOMS{A \sqcap B} &= \ATOMS{A} \cup \ATOMS{B}
\end{align*}

The consistency of two types is determined by the consistency of its
atoms. \\

\noindent \textbf{Lemma} (Atomic Consistency) 
\begin{enumerate}
\item If $A \sim B$, $C \in \ATOMS{A}$, and $D \in \ATOMS{B}$,
  then $C \sim D$.
\item If (for any $C \in \ATOMS{A}$ and $D \in \ATOMS{B}$, 
  $C \sim D$), then $A \sim B$.
\item If $A \not\sim B$, then $C \not\sim D$ for some
  $C \in \ATOMS{A}$ and $D \in \ATOMS{B}$.
\item If $C \not\sim D$, $C \in \ATOMS{A}$, and $D \in \ATOMS{B}$,
  then $A \not\sim B$. 
\end{enumerate}

There are also several properties of subtyping and the atoms of a type.\\

\noindent \textbf{Lemma} (Atomic Subtyping)
\begin{enumerate}
\item If $A <: B$ and $C \in \ATOMS{B}$, then $A <: C$.
\item If $A <: n$, then $n \in \ATOMS{A}$.
\item $n <: A$ if and only if $\ATOMS{A} \subseteq \{ n \}$.
\item If $C <: A \to B$, then $D\to E \in \ATOMS{C}$ for some $D,E$.
\item If $\Gamma \vdash A$ and every atom in $\Gamma$ is a function type,
  then every atom of $A$ is a function type.
\end{enumerate}

And we have the following important inversion lemma for function
types. We use the following abbreviations:
\begin{align*}
  \mathrm{dom}(\Gamma) &= \{ A \mid \exists B.\; A \to B \in \Gamma \}\\
  \mathrm{cod}(\Gamma) &= \{ B \mid \exists A.\; A \to B \in \Gamma \}
\end{align*}

\noindent \textbf{Lemma} (Subtyping Inversion for Function Types)\\
%
If $C <: A \to B$, then there is a sequence of function types $\Gamma$
such that
\begin{enumerate}
\item each element of $\Gamma$ is an atom of $C$,
\item For every $D\to E \in \Gamma$, we have $A <: D$, and
\item $\sqinter \mathrm{cod}(\Gamma) <: B$.
\end{enumerate}
Note that item 2 above implies that $A <: \sqinter \mathrm{dom}(\Gamma)$.

\noindent \textbf{Lemma} (Consistency and Subtyping)
\begin{enumerate}
\item  If $A \sim B$, $A <: C$, and $B <: D$,
  then $C \sim D$.
\item If $A \not\sim B$, $C <: A$, $D <: B$, then $C \not\sim D$.
\end{enumerate}

(1) The proof is by strong induction on the sum of the depths of $A$,
$B$, $C$, and $D$. We define the depth of a type as follows.
\begin{align*}
  \mathit{depth}(n) &= 0 \\
  \mathit{depth}(A \to B) &= 1 + \mathrm{max}(\mathit{depth}(A),\mathit{depth}(B)) \\
  \mathit{depth}(A \sqcap B) &= \mathrm{max}(\mathit{depth}(A),\mathit{depth}(B)) 
\end{align*}
To show that $C \sim D$ it suffices to show that
all of their atoms are consistent. Suppose $C' \in \ATOMS{C}$
and $D'\in\ATOMS{D}$. So we need to show that $C' \sim D'$.
%Note that $A <: C'$ and $B <: D'$
We proceed by cases on $C'$.
\begin{itemize}
\item Case $C'=n_1$:\\
  We have $A <: C'$ and therefore $n_1 \in \ATOMS{A}$.
  Then because $A \sim B$, we have $\ATOMS{B} \subseteq \{n_1\}$.
  We have $B <: D'$, so we also have $\ATOMS{D} \subseteq \{n_1\}$.
  Therefore $C' \sim D'$.
  
\item Case $C'=C_1\to C_2$:\\
  We have $A <: C_1 \to C_2$, so by inversion we have
  some sequence of function types $\Gamma_1$ such that
  every element of $\Gamma_1$ is an atom of $A$,
  $C_1 <: \sqinter \mathrm{dom}(\Gamma_1)$,
  and $\sqinter \mathrm{cod}(\Gamma_1) <: C_2$.

  We also know that $D'$ is a function type, say $D'=D_1 \to D_2$.
  (This is because we have $A <: C'$, so we know that $A_1\to A_2 \in
  \ATOMS{A}$ for some $A_1,A_2$. Then because $A \sim B$, we know that
  all the atoms in $B$ are function types.  Then because $B <: D$ and
  $D' \in \ATOMS{D}$, we have that $D'$ is a function type.)
  %
  So by inversion on $B <: D_1 \to D_2$, we have
  some sequence of function types $\Gamma_2$ such that
  every element of $\Gamma_2$ is an atom of $B$,
  $D_1 <: \sqinter \mathrm{dom}(\Gamma_2)$,
  and $\sqinter \mathrm{cod}(\Gamma_2) <: D_2$.

  It's the case that either $C_1 \sim D_1$ or $C_1 \not\sim D_1$.
  \begin{itemize}
  \item Sub-case $C_1 \sim D_1$.\\
    It suffices to show that $C_2 \sim D_2$.
    By the induction hypothesis, we have
    $\sqinter \mathrm{dom}(\Gamma_1) \sim \sqinter \mathrm{dom}(\Gamma_2)$.

    As an intermediate step, we shall prove that
    $\sqinter \mathrm{cod}(\Gamma_1) \sim \sqinter \mathrm{cod}(\Gamma_2)$,
    which we shall do by showing that all their atoms are consistent.
    Suppose $A' \in \ATOMS{\sqinter \mathrm{cod}(\Gamma_1)}$
    and $B' \in \ATOMS{\sqinter \mathrm{cod}(\Gamma_2)}$.
    There is some $A_1\to A_2 \in \Gamma_1$ where $A' \in \ATOMS{A_2}$.
    Similarly, there is $B_1 \to B_2 \in \Gamma_2$ where $B' \in \ATOMS{B_2}$.
    Also, we have $A_1 \to A_2 \in \ATOMS{A}$ and $B_1 \to B_2 \in \ATOMS{B}$.
    Then because $A \sim B$, we have $A_1 \to A_2 \sim B_1 \to B_2$.
    Furthermore, we have $A_1 \sim B_1$ because
    $\sqinter \mathrm{dom}(\Gamma_1) \sim \sqinter \mathrm{dom}(\Gamma_2)$,
    so it must be the case that $A_2 \sim B_2$.
    Then because $A' \in \ATOMS{A_2}$ and $B' \in \ATOMS{B_2}$, we
    have $A' \sim B'$. Thus concludes this intermediate step.

    By another use of the induction hypothesis, we have
    $C_2 \sim D_2$, and this case is finished.
    
  \item Sub-case $C_1 \not\sim D_1$.\\
    Then we immediately have $C_1 \to C_2 \sim D_1 \to D_2$.
  \end{itemize}
  
\item Case $C'=C_1\sqcap C_2$:\\
  We already know that $C'$ is an atom, so we have a contradiction
  and this case is vacuously true.
\end{itemize}

The next two lemmas follow from the Consistency and Subtyping Lemma
and help prepare to prove the case for application in the Join
Theorem. \\

\noindent \textbf{Lemma} (Application Consistency) \\
%
If $A_1 \sim A_2$, $B_1 \sim B_2$, $A_1 <: B_1 \to C_1$,
$A_2 <: B_2 \to C_2$, and all these types are well formed,
then $C_1 \sim C_2$.\\
(This lemma is proved directly, without induction.)\\

\noindent \textbf{Lemma} (Application Intersection) \\
%
If $A_1 <: B_1 \to C_1$, $A_2 <: B_2 \to C_2$, $A_1 \sim A_2$,
$B_1 \sim B_2$, and $C_1 \sim C_2$, then
$(A_1\sqcap A_2) <: (B_1 \sqcap B_2) \to (C_1 \sqcap C_2)$.\\
(This lemma is proved directly, without induction.) \\


\section{Updating the Denotational Semantics}

Armed with the Consistency and Subtyping Lemma, I turned back to the
proof of the Join Theorem, but first I needed to update my
denotational semantics to use intersection types instead of
values. For this we'll need the definition of well formed types that
we alluded to earlier.

\begin{gather*}
  \frac{}{\WF{n}}
  \qquad
  \frac{\WF{A} \quad \WF{B}}{\WF{A \to B}}
  \qquad
  \frac{A \sim B \quad \WF{A} \quad \WF{B}}{\WF{A \sqcap B}}
\end{gather*}

Here are some examples and non-examples of well-formed types.
\begin{gather*}
  \WF{4} \qquad \WF{3 \sqcap 3} \qquad \neg \WF{3 \sqcap 4} \\
  \WF{(0\to 1) \sqcap (2 \to 3)} \qquad \neg \WF{(0 \to 1) \sqcap (0 \to 2)}
\end{gather*}
It is sometimes helpful to think of well-formed types in terms of the
equivalence classes determined by subtype equivalence:
\[
A \approx B \quad = \quad
  A <: B \text{ and } B <: A
\]
For example, we have $3 \approx (3 \sqcap 3)$, so they are
in the same equivalence class and $3$ would be the representative.


We also introduce the following notation for all the well-formed types
that are super-types of a given type.
\[
   \UP{A} \quad = \quad \{ B\mid A <: B \text{ and } \WF{B} \}
\]

We shall represent variables with de Bruijn indices, so an environment
$\Gamma$ is a sequence of types.  The denotational semantics of the
call-by-value $\lambda$-calculus is defined as follows.
\begin{align*}
  \ESEM{n}\Gamma &= \UP{n} \\
  \ESEM{x}\Gamma &=  \IF x < |\Gamma| \THEN \UP{\Gamma[k]} \ELSE \emptyset \\
  \ESEM{\lambda e }\Gamma &= \{ A \mid \WF{A} \text{ and } \FSEM{A}e\Gamma \} \\
  \ESEM{e_1\app e_2}\Gamma &= \left\{ C\, \middle| 
      \begin{array}{l}
      \exists A,B.\; A \in \ESEM{e_1}\Gamma,
      B \in \ESEM{e_2}\Gamma,\\
      A <: B \to C, \text{ and } \WF{C}
      \end{array}
        \right\} \\
  \ESEM{f(e_1,e_2)}\Gamma &=
      \left\{ C\, \middle| \begin{array}{l}
       \exists A,B,n_1,n_2.\; A \in \ESEM{e_1}\Gamma, B \in \ESEM{e_2}\Gamma,\\
      A <: n_1, B <: n_2, \SEM{f}(n_1,n_2) <: C, \WF{C} 
      \end{array} \right\} \\
  \ESEM{\IF e_1 \THEN e_2 \ELSE e_3}\Gamma &=
    \left\{ B\, \middle| 
    \begin{array}{l}\exists A, n.\; A \in \ESEM{e_1}\Gamma, A <: n,\\
           n = 0 \Rightarrow B \in \ESEM{e_3}\Gamma,\\
           n \neq 0 \Rightarrow B \in \ESEM{e_2}\Gamma
    \end{array}
    \right\}
    \\[2ex]
   \FSEM{n}e\Gamma &= \mathit{false} \\
   \FSEM{A \sqcap B}e \Gamma &= \FSEM{A}e\Gamma \text{ and } \FSEM{B}e\Gamma\\
   \FSEM{A \to B}e \Gamma &= B \in \ESEM{e} (A, \Gamma)
\end{align*}

It is easy to show that swapping in a ``super'' environment does not
change the semantics.

\noindent \textbf{Lemma} (Weakening)
\begin{enumerate}
\item If $\FSEM{A}e \Gamma_1$, $\Gamma_1 <: \Gamma_2$ and
  $(\forall B, \Gamma_1, \Gamma_2.\;
     B \in \ESEM{e}\Gamma_1, \Gamma_2 <: \Gamma_1
    \Rightarrow B \in \ESEM{e}\Gamma_2)$, then
  $\FSEM{A}e \Gamma_2$.
\item If $A \in \ESEM{e}\Gamma_1$ and $\Gamma_2 <: \Gamma_1$,
  then $A \in \ESEM{e}\Gamma_2$.
\end{enumerate}
(Part 1 is proved by induction on $A$. Part 2 is proved by induction
on $e$ and uses part 1.) \\

\section{The Home Stretch}

Now for the main event, the proof of the Meet Theorem! \\

\noindent \textbf{Theorem} (Meet)\\
If $A_1 \in \ESEM{e}\Gamma_1$, $A_2 \in \ESEM{e}\Gamma_2$,
both $\Gamma_1$ and $\Gamma_2$ are well formed,
and $\Gamma_1 \sim \Gamma_2$, \\
then $A_1 \sqcap A_2 \in \ESEM{e}(\Gamma_1\sqcap\Gamma_2)$ and
  $\WF{A_1 \sqcap A_2}$. \\[1ex]
%
\textbf{Proof}
We proceed by induction on $e$.
\begin{itemize}
\item Case $e=k$ ($k$ is a de Bruijn index for a variable):\\
  We have $\Gamma_1[k] <: A_1$ and $\Gamma_2[k] <: A_2$,
  so $\Gamma_1[k] \sqcap \Gamma_2[k] <: A_1 \sqcap A_2$.
  Also, because $\Gamma_1 \sim \Gamma_2$ we have
  $\Gamma_1[k] \sim \Gamma_2[k]$ and therefore
  $A_1 \sim A_2$, by the Consistency and Subtyping Lemma.
  So we have $\WF{A_1 \sqcap A_2}$ and this case is finished.
  
\item Case $e=n$: \\
  We have $n <: A_1$ and $n <: A_2$, so $n <: A_1 \sqcap A_2$.
  Also, we have $A_1 \sim A_2$ by the Consistency and Subtyping Lemma.
  So we have $\WF{A_1 \sqcap A_2}$ and this case is finished.

\item Case $e=\lambda e$: \\
  We need to show that $\WF{A_1 \sqcap A_2}$ and
  $\FSEM{A_1 \sqcap A_2}e(\Gamma_1\sqcap\Gamma_2)$.
  For the later, it suffices to show that $A_1 \sim A_2$,
  which we shall do by showing that their atoms are consistent.
  Suppose $A'_1 \in \ATOMS{A_1}$ and $A'_2 \in \ATOMS{A_2}$.
  Because $\FSEM{A_1}e\Gamma_1$ 
  we have $A'_1 =A'_{11} \to A'_{12}$ and
  $A'_{12} \in \ESEM{e}(A'_{11},\Gamma_1)$. Similarly,
  from $\FSEM{A_2}e\Gamma_2$ we have
  $A'_2 =A'_{21} \to A'_{22}$ and $A'_{22} \in \ESEM{e}(A'_{21},\Gamma_2)$.
  We proceed by cases on whether $A'_{11} \sim A'_{21}$.
  \begin{itemize}
  \item Sub-case $A'_{11} \sim A'_{21}$:\\
    By the induction hypothesis, we have
    $\WF{A'_{12} \sqcap A'_{22}}$ from which we
    have $A'_{12} \sim A'_{22}$ and therefore
    $A'_{11}\to A'_{12} \sim A'_{21} \to A'_{22}$.
    
    %% \[
    %%   A'_{12} \sqcap A'_{22} \in
    %%   \ESEM{e}(A'_{11}\sqcap A'_{21},\Gamma_1 \sqcap \Gamma_2)
    %%   \quad\text{and} \quad
    %%   \WF{A'_{12} \sqcap A'_{22}}
    %% \]
    
    
  \item Sub-case $A'_{11} \not\sim A'_{21}$:\\
      It immediately follows that $A'_{11}\to A'_{12} \sim A'_{21} \to A'_{22}$.
  \end{itemize}

  It remains to show $\FSEM{A_1 \sqcap A_2}e(\Gamma_1\sqcap\Gamma_2)$.
  This follows from two uses of the Weakening Lemma to obtain
  $\FSEM{A_1}e(\Gamma_1\sqcap\Gamma_2)$ and
  $\FSEM{A_2}e(\Gamma_1\sqcap\Gamma_2)$.
  
\item Case $e = (e_1 \app e_2)$: \\
  We have
  \[
  B_1 \in \ESEM{e_1}\Gamma_1 \quad
  C_1 \in \ESEM{e_2}\Gamma_1 \quad
  B_1 <: C_1 \to A_1 \quad
  \WF{A_1}
  \]
  and
  \[
  B_2 \in \ESEM{e_1}\Gamma_2 \quad
  C_2 \in \ESEM{e_2}\Gamma_2 \quad
  B_2 <: C_2 \to A_2 \quad
  \WF{A_2}
  \]
  By the induction hypothesis, we have
  \[
  B_1 \sqcap B_2 \in \ESEM{e_1}(\Gamma_1 \sqcap \Gamma_2)
  \quad
  \WF{B_1 \sqcap B_2}
  \]
  and
  \[
  C_1 \sqcap C_2 \in \ESEM{e_2}(\Gamma_1 \sqcap \Gamma_2)
  \quad
  \WF{C_1 \sqcap C_2}
  \]
  We obtain $A_1 \sim A_2$ by the Application Consistency Lemma,
  and then by the Application Intersection Lemma we have
  \[
    B_1 \sqcap B_2 <: (C_1 \sqcap C_2) \to (A_1 \sqcap A_2)
  \]
  So we have $A_1 \sqcap A_2 \in \ESEM{e}(\Gamma_1 \sqcap \Gamma_2)$.
    
  Also, from $A_1 \sim A_2$, $\WF{A_1}$, and $\WF{A_2}$, we conclude
  that $\WF{A_1 \sqcap A_2}$.
\item Case $e= f(e_1,e_2)$: \\
  (This case is not very interesting. See the Isabelle proof
   for the details.)

 \item Case $e= \IF e_1 \THEN e_2 \ELSE e_3$: \\
  (This case is not very interesting. See the Isabelle proof
   for the details.)
  
\end{itemize}

I thought that the following Subsumption Theorem would be needed to
prove the Meet Theorem, but it turned out not to be necessary, which
is especially nice because the proof of the Subsumption Theorem turned
out to depend on the Meet Theorem!\\

\noindent \textbf{Theorem} (Subsumption)\\
If $A \in \ESEM{e}\Gamma$, $A <: B$, and both $B$ and $\Gamma$ are
well-formed, then $B \in \ESEM{e}\Gamma$. \\

\noindent The proof is by induction on $e$ and all but the case
$e=\lambda e'$ are straightforward. For that case, we use the
following lemmas.\\

\noindent \textbf{Lemma} (Distributivity for $\mathcal{F}$)\\
If $\FSEM{(A \to B)\sqcap (C \to D)} e \Gamma$,
$A \sim C$, and everything is well formed, then
$\FSEM{(A\sqcap C) \to (B\sqcap D)} e \Gamma$.\\
(The proof is direct, using the Meet Theorem
and the Weakening Lemma.)\\

\noindent \textbf{Lemma} ($\mathcal{F}$ and Intersections)\\
Suppose $\Gamma_1$ is a non-empty sequence of well-formed and
consistent function types.
If $\FSEM{\sqinter \Gamma_1} e \Gamma_2$, then
$\FSEM{\sqinter\mathrm{dom}(\Gamma_1) \to \sqinter \mathrm{cod}(\Gamma_1)}
e \Gamma_2$. \\
(The proof is by induction on $\Gamma_1$ and uses the previous lemma.)

\section{Conclusion}

This result can be viewed a couple ways. As discussed at the beginning
of this post, establishing the Meet Theorem means that the this
call-by-value denotational semantics respects $\beta$-equality for any
terminating argument expression. This is useful in proving the
correctness of a function inlining optimizer. Also, it would be
straightforward to define a call-by-name (or need) version of the
semantics that respects $\beta$-equality unconditionally.

Secondly, from the viewpoint of intersection type systems, this result
shows that, once we require types to be well formed (i.e. self
consistent), we no longer need the intersection introduction rule
because it is a consequence of having the subtyping rule for
distributing intersections through function types.



\end{document}

%%  LocalWords:  denotationally ctx subexpression eq denot lrcl env
%%  LocalWords:  iff Hah pdf subtyping contra distributivity BCD et
%%  LocalWords:  Barendregt al Fundamenta Informaticae sequent Coq th
%%  LocalWords:  FLOC Scott's Olivier's disjunction reflexivity dom
%%  LocalWords:  subtype Denotational denotational de Bruijn indices
%%  LocalWords:  Subsumption inlining
