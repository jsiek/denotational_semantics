\documentclass[11pt,a4paper]{article}
\usepackage{isabelle,isabellesym}
\usepackage{fullpage}

% further packages required for unusual symbols (see also
% isabellesym.sty), use only when needed

%\usepackage{amssymb}
  %for \<leadsto>, \<box>, \<diamond>, \<sqsupset>, \<mho>, \<Join>,
  %\<lhd>, \<lesssim>, \<greatersim>, \<lessapprox>, \<greaterapprox>,
  %\<triangleq>, \<yen>, \<lozenge>

%\usepackage{eurosym}
  %for \<euro>

\usepackage[only,bigsqcap]{stmaryrd}
  %for \<Sqinter>

%\usepackage{eufrak}
  %for \<AA> ... \<ZZ>, \<aa> ... \<zz> (also included in amssymb)

%\usepackage{textcomp}
  %for \<onequarter>, \<onehalf>, \<threequarters>, \<degree>, \<cent>,
  %\<currency>

% this should be the last package used
\usepackage{pdfsetup}

% urls in roman style, theory text in math-similar italics
\urlstyle{rm}
\isabellestyle{it}

% for uniform font size
%\renewcommand{\isastyle}{\isastyleminor}


\begin{document}

\title{Modeling the Functional Character \\ of the Lambda Calculus}
\author{Jeremy G. Siek}
\maketitle

\begin{abstract}
  Denotational semantics for the lambda calculus that are given by graph models
  or filter models (i.e. intersection type systems) use finite relations
  to model functions. In this document I use a notion of consistency 
  to restrict these relations to be functions (or finite approximations of them) 
  and I prove that the model is deterministic in the sense that any two outputs
  from the same program are equal, and furthermore, that the result of merging
  the two outputs is also an output of the program. Phrasing this in terms
  of an intersection type system, I show that the subsumption rule and the
  intersection introduction rule are both admissible.
\end{abstract}

\tableofcontents

% sane default for proof documents
\parindent 0pt\parskip 0.5ex

\pagebreak

% generated text of all theories
\input{session}

% optional bibliography
%\bibliographystyle{abbrv}
%\bibliography{root}

\end{document}

%%% Local Variables:
%%% mode: latex
%%% TeX-master: t
%%% End:
