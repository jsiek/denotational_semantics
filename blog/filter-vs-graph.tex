\documentclass{article}
%\usepackage{cmbright}
\usepackage{natbib}
\usepackage{amsthm}
\usepackage{amsmath}
\usepackage{amssymb}
%\usepackage{mathabx}
\usepackage{stmaryrd}
\usepackage{semantic}
%\usepackage{fullpage}
%\usepackage{fontspec,unicode-math}
\usepackage{hyperref}

\newtheorem{theorem}{Theorem}
\newtheorem{lemma}[theorem]{Lemma}
\newtheorem{corollary}[theorem]{Corollary}
\newtheorem{proposition}[theorem]{Proposition}
\newtheorem{constraint}[theorem]{Constraint}
\newtheorem{definition}[theorem]{Definition}
\newtheorem{example}[theorem]{Example}

\newcommand{\true}[0]{\mathsf{true}}
\newcommand{\curry}[1]{\mathsf{curry}(#1)}
\newcommand{\apply}[2]{\mathsf{apply}(#1, #2)}

\title{Filter versus Graph Models}
\author{Jeremy G. Siek}

\begin{document}
\maketitle

Filter models and graph models are two different ways to give a
denotational semantics to the lambda calculus. At first glance they
appear quite different, but many of those differences do not matter.
On the other hand, there are some differences that do matter.
Differences that matter are ones that cause significant differences in
the proofs about the semantics, such as proofs of soundness and
adequacy with respect to the reduction semantics.










\bibliographystyle{abbrvnat}
\bibliography{all}

\end{document}
