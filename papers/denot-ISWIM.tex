\documentclass{article}
%\usepackage{cmbright}
\usepackage{natbib}
\usepackage{amsthm}
\usepackage{amsmath}
\usepackage{amssymb}
%\usepackage{mathabx}
\usepackage{stmaryrd}
\usepackage{semantic}
%\usepackage{fullpage}
%\usepackage{fontspec,unicode-math}
\usepackage{hyperref}

\newtheorem{theorem}{Theorem}
\newtheorem{lemma}[theorem]{Lemma}
\newtheorem{corollary}[theorem]{Corollary}
\newtheorem{proposition}[theorem]{Proposition}
\newtheorem{constraint}[theorem]{Constraint}
\newtheorem{definition}[theorem]{Definition}
\newtheorem{example}[theorem]{Example}

\newcommand{\true}[0]{\mathsf{true}}
\newcommand{\curry}[1]{\mathsf{curry}(#1)}
\newcommand{\apply}[2]{\mathsf{apply}(#1, #2)}

\title{An Elementary Denotational \\ Semantics for ISWIM}
\author{Jeremy G. Siek}

\begin{document}
\maketitle

\begin{abstract}
  This article takes a baby step toward developing denotational
  semantics that are suitable for widespread use by programming
  language practitioners and researchers who work outside of
  semantics, e.g, in compiler correctness, static analysis, language
  standards, etc. The current practice in such areas is to use
  operational semantics, and it is quite successful. However, such
  semantics are not compositional, which complicates reasoning about
  them.  Denotational semantics are compositional, but their
  mathematical sophistication has made them too time consuming for
  practitioners to understand and customize to their uses. One purpose
  of this article is to see how little mathematics is necessary to
  develop compositional semantics.

  As a case study, this article develops a denotational semantics for
  the ISWIM programming language, that is, for a call-by-value lambda
  calculus extended with arbitrary primitive operations and literals.
  It models lambda expressions with lookup tables, a simple idea that
  goes back to the graph and filter models for the lambda
  calculus. Unfortunately, those models cannot be used as-is for
  ISWIM: lookup tables are too loose in that they can represent
  arbitrary relations, not just functions. So the article presents a
  way to restrict lookup tables to just functions. The article proves
  the soundness and adequacy of the resulting denotational semantics,
  proves that denotational equality implies contextual equivalence,
  and demostrates uses of the semantics in compiler correctness proofs
  (inlining and closure conversion).
\end{abstract}


\section{Introduction}

\citet{Barendregt:2013aa}

\section{Semantics of ISWIM}

Values
\[
\begin{array}{lcll}
  \mathbb{A} & ::= & \mathbb{Z} \mid \mathbb{B} \mid \cdots & \text{atomic types}\\
  \mathbb{P} & ::= & \mathbb{A} \mid \mathbb{A} \to \mathbb{P} & \text{primitive types}\\
  c & \in & \mathbb{P} & \text{constants} \\
  u,v,w & ::= & c \mid \bot \mid v \mapsto w \mid u \sqcup v & \text{values}
\end{array}
\]

\begin{align*}
  \curry{D}\, \gamma\, (v \mapsto w) &= D \,(\gamma,v)\, w\\
  \curry{D}\, \gamma\, (u \sqcup v) &= D \,\gamma\, u \text{ and } D\, \gamma\, v \\
  \curry{D}\, \gamma\, \bot &= \true \\
\end{align*}


\begin{align*}
  \apply{D}{E}\,\gamma\,w &=
    \exists v.\, D\,\gamma\,(v \mapsto w) \text{ and } E \, \gamma \, v
\end{align*}

\section{Conclusion}


\end{document}
