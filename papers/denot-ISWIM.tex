\documentclass{article}
%\usepackage{cmbright}
\usepackage{natbib}
\usepackage{amsthm}
\usepackage{amsmath}
\usepackage{amssymb}
%\usepackage{mathabx}
\usepackage{stmaryrd}
\usepackage{semantic}
%\usepackage{fullpage}
%\usepackage{fontspec,unicode-math}
\usepackage{hyperref}

\newtheorem{theorem}{Theorem}
\newtheorem{lemma}[theorem]{Lemma}
\newtheorem{corollary}[theorem]{Corollary}
\newtheorem{proposition}[theorem]{Proposition}
\newtheorem{constraint}[theorem]{Constraint}
\newtheorem{definition}[theorem]{Definition}
\newtheorem{example}[theorem]{Example}

\newcommand{\true}[0]{\mathsf{true}}
\newcommand{\curry}[1]{\mathsf{curry}(#1)}
\newcommand{\apply}[2]{\mathsf{apply}(#1, #2)}

\title{An Elementary Denotational \\ Semantics for ISWIM}
\author{Jeremy G. Siek}

\begin{document}
\maketitle

\begin{abstract}
  This article takes a small step toward developing denotational
  semantics that are suitable for widespread use by programming
  language practitioners and researchers who work outside of
  semantics, e.g, in compiler correctness, static analysis, language
  standards, etc. The current practice in such areas is to use
  operational semantics, and it is quite successful. However, such
  semantics are not compositional, which complicates reasoning about
  them.  Denotational semantics are compositional, but their
  mathematical sophistication has made them too time consuming for
  practitioners to understand and customize to their uses. One purpose
  of this article is to see how little sophistication and abstraction
  is necessary to define a compositional semantics.

  As a case study, this article develops a denotational semantics for
  the ISWIM programming language, that is, for a call-by-value lambda
  calculus extended with arbitrary primitive operations and literals.
  We choose a denotational model based on the graph and filter models
  for the lambda calculus because of their extreme simplicity; they
  use lookup tables to represent functions. Unfortunately, these
  models are not adequate as-is for ISWIM because lookup tables
  represent arbitrary relations, not just functions. This article
  presents a simple way to restrict lookup tables so that they only
  approximate functions. The article proves the soundness and adequacy
  of the resulting denotational semantics for ISWIM, proves that
  denotational equality implies contextual equivalence, and
  demostrates uses of the semantics in compiler correctness proofs
  (inlining and closure conversion).
\end{abstract}


\section{Introduction}

\citet{Barendregt:2013aa}

\section{Semantics of ISWIM}

Values
\[
\begin{array}{lcll}
  \mathbb{A} & ::= & \mathbb{Z} \mid \mathbb{B} \mid \cdots & \text{atomic types}\\
  \mathbb{P} & ::= & \mathbb{A} \mid \mathbb{A} \to \mathbb{P} & \text{primitive types}\\
  c & \in & \mathbb{P} & \text{constants} \\
  u,v,w & ::= & c \mid \bot \mid v \mapsto w \mid u \sqcup v & \text{values}
\end{array}
\]

\begin{align*}
  \curry{D}\, \gamma\, (v \mapsto w) &= D \,(\gamma,v)\, w\\
  \curry{D}\, \gamma\, (u \sqcup v) &= D \,\gamma\, u \text{ and } D\, \gamma\, v \\
  \curry{D}\, \gamma\, \bot &= \true \\
\end{align*}


\begin{align*}
  \apply{D}{E}\,\gamma\,w &=
    \exists v.\, D\,\gamma\,(v \mapsto w) \text{ and } E \, \gamma \, v
\end{align*}

\section{Conclusion}

\bibliographystyle{abbrvnat}
\bibliography{all}

\end{document}
