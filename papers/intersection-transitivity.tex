\documentclass{article}
\usepackage{natbib}
\usepackage{amsthm}
\usepackage{amsmath}
\usepackage{amssymb}
\usepackage{mathabx}
\usepackage{stmaryrd}
\usepackage{semantic}
%\usepackage{fullpage}
\usepackage{fontspec,unicode-math}

\newtheorem{theorem}{Theorem}
\newtheorem{lemma}[theorem]{Lemma}
\newtheorem{corollary}[theorem]{Corollary}
\newtheorem{proposition}[theorem]{Proposition}
\newtheorem{constraint}[theorem]{Constraint}
\newtheorem{definition}[theorem]{Definition}
\newtheorem{example}[theorem]{Example}


\title{Transitivity of Subtyping for Intersection Types}
\author{Jeremy G. Siek}

\begin{document}
\maketitle

\newcommand{\TOP}{\ensuremath{\mathtt{U}}}
\newcommand{\dom}[1]{\mathrm{dom}(#1)}
\newcommand{\cod}[1]{\mathrm{cod}(#1)}


\begin{abstract}
  The subtyping relation for intersection type systems traditionally
  employs a transitivity rule (Barendregt et al. 1983), which means
  that the subtyping judgment does not enjoy the subformula property.
  Laurent develops a sequent-style subtyping judgment, without
  transitivity, and proves transitivity via a sequence of six lemmas
  that culminate in cut-elimination (2018). This article presents a
  subtyping judgment, in regular style, that satisfies the subformula
  property, and presents a direct proof of transitivity. Borrowing
  from Laurent's system, the rule for function types is essentially
  the $\beta$-soundness property.  The main lemma required for the
  transitivity proof happens to be one that is used to prove the
  inversion principle for subtyping of function types (Barendregt et
  al. 2013). However, the choice of induction principle for the proof
  of transitivity requires some care: we use strong induction on the
  lexicographical ordering of the sum of the depths of the first and
  last type followed by the sume of the sizes of the first and second
  type.
\end{abstract}

\section{Introduction}

Intersection types were invented by Coppo, Dezani-Ciancaglini, and
Salle, as a tool for studying normalization in the lambda
calculus~\citet{Coppo:1979aa}. By varying the subtyping rules and atom
types, researchers have used intersection type systems to model many
different
calculi~\citep{Coppo:1980ab,Coppo:1981aa,Engeler:1981aa,Coppo:1984aa,Honsell:1992aa,Abramsky:1993fk,Plotkin:1993ab,Honsell:1999aa,Rocca:2004aa,Dezani-Ciancaglini:2005aa,Alessi:2006aa}.
Perhaps the best-known of them is the BCD intersection type system of
\citet{Barendregt:1983aa}.


\citet{Barendregt:2013aa}

\citet{Laurent:2012aa}
\citet{Laurent:2018aa}

%$λ^{\mathrm{BCD}}_∩$

Figure~\ref{fig:types}

\begin{figure}[tbp]
  \[
  \begin{array}{lclr}
    \alpha,\beta & ::= & \TOP \mid c_0 \mid c_1 \mid c_2 \mid \cdots & \text{atoms}\\
    A,B,C,D & ::= & \alpha \mid A → B \mid A ∩ B & \text{types}
  \end{array}
  \]
  \caption{Intersection Types}
  \label{fig:types}
\end{figure}



Figure~\ref{fig:BCD-subtyping}

\begin{figure}[tbp]
  \fbox{$A ≤ B$} \\[1ex]
  
  \centering
  \begin{tabular}{p{1in}l}
    (refl)  & $A ≤ A$ \\[3ex]
    (incl$_L$) & $A ∩ B ≤ A$ \\[3ex]
    (incl$_R$) & $A ∩ B ≤ B$ \\[3ex]
    (glb) & $\inference{C ≤ A & C ≤ B}{C ≤ A ∩ B}$ \\[3ex]
    (trans) & $\inference{A ≤ B & B ≤ C}{A ≤ C}$ \\[3ex]
    ($\TOP_{\mathrm{top}}$) & $A ≤ \TOP$ \\[3ex]
    ($\TOP{→}$) & $\TOP ≤ A → \TOP$ \\[3ex]
    (${→}{∩}$) & $(A → B) ∩ (A → C) ≤ A → (B ∩ C)$ \\[3ex]
    ($→$) & $\inference{A' ≤ A & B ≤ B'}{A → B ≤ A' → B'}$
  \end{tabular}
  \caption{The Subtyping Judgement of Barendregt, Coppo, and Dezani (BCD).}
  \label{fig:BCD-subtyping}
\end{figure}

Figure~\ref{fig:aux}

\begin{figure}[tbp]

  \fbox{$\dom{A}, \cod{A}$}
  \begin{align*}
  \dom{A → B} &= A \\
  \dom{A ∩ B} &= \dom{A} ∩ \dom {B} \\
  \\
  \cod{A → B} &= B \\
  \cod{A ∩ B} &= \cod{A} ∩ \cod {B}
  \end{align*}

  \fbox{$\mathsf{top}(A)$}
  \begin{gather*}
    \mathsf{top}(\TOP)
    \quad
    \inference{\mathsf{top}(B)}{\mathsf{top}(A → B)}
    \quad
    \inference{\mathsf{top}(A) & \mathsf{top}(B)}{\mathsf{top}(A ∩ B)}
  \end{gather*}

  \fbox{$A ∈ B$}
  \begin{gather*}
    \inference{}{\alpha ∈ \alpha}  \quad
    \inference{}{A → B ∈ A → B} \\[3ex]
    \inference{A ∈ B}{A ∈ B ∩ C} \quad
    \inference{A ∈ C}{A ∈ B ∩ C}
  \end{gather*}

  \fbox{$A ⊆ B$}
  \[
     A ⊆ B = ∀ C.\, C ∈ A \text{ implies } C ∈ B
  \]

  \fbox{$\mathsf{topInCod}(D)$}
  \[
  \mathsf{topInCod}(D) =
     \exists A B.\, A → B ∈ D \text{ and } \mathsf{top}(B)  
  \]

  \caption{Auxiliary definitions}
  \label{fig:aux}
\end{figure}


\begin{figure}[tbp]
  \fbox{$A <: B$} \\[1ex]
  
  \centering
  \begin{tabular}{p{1in}l}
    (refl$_α$) & $\inference{}{α <: α}$ \\[3ex]
    (lb$_L$) & $\inference{A <: C}{A ∩ B <: C}$ \\[3ex]
    (lb$_R$) & $\inference{B <: C}{A ∩ B <: C}$ \\[3ex]
    (glb) & $\inference{C ≤ A & C ≤ B}{C ≤ A ∩ B}$ \\[3ex]
    ($\TOP_{\mathrm{top}}$) & $\inference{}{A <: \TOP}$ \\[3ex]
    ($\TOP{→}'$) & $\inference{}{C <: A → B}\;\mathsf{top}(B)$ \\[3ex]
    ($→'$) &$\inference{A <: \dom{D} & \cod{D} <: B }{C <: A → B}\begin{array}{l} \neg\, \mathsf{top}(B) \\ \neg \, \mathsf{topInCod}(D) \\ D ⊆ C \end{array}$
  \end{tabular}
  \caption{New Subtyping Relation}
  \label{fig:new-subtyping}
\end{figure}

\begin{definition}[factors]
  We say $A → B$ \emph{factors} $C$
  if there exists some type $C'$ such that
  $C' ⊆ C$, $\neg\,\mathsf{topInCod}(C')$, $A <: \dom{C'}$, and $\cod{C'} <: B$.
\end{definition}

\begin{lemma}\label{lem:sub-fun-inv}
  If $\neg\,\mathsf{top}(B)$,
  $D <: C$, and
  $A → B ∈ C$, then
  $A → B$ factors $D$.
\end{lemma}


\begin{lemma}\label{lem:sub-inv-trans}
  If
  \begin{itemize}
  \item $\neg\, \mathsf{topInCod}(D)$,
  \item for any $A,B$, if $\neg\,\mathsf{top}(B)$ and $A → B ∈ D$,
    then $A → B$ factors $C$,
  \end{itemize}
  then $\dom{D} → \cod{D}$ factors $C$.
\end{lemma}

\begin{align*}
  \mathsf{size}(\alpha) &= 0 \\
  \mathsf{size}(A → B) &= 1 + \mathsf{size}(A) + \mathsf{size}(B) \\
  \mathsf{size}(A ∩ B) &= 1 + \mathsf{size}(A) + \mathsf{size}(B)
\end{align*}

\begin{align*}
  \mathsf{depth}(\alpha) &= 0 \\
  \mathsf{depth}(A → B) &= 1 + \max(\mathsf{depth}(A), \mathsf{depth}(B)) \\
  \mathsf{depth}(A ∩ B) &= \max(\mathsf{depth}(A), \mathsf{depth}(B))
\end{align*}

\begin{lemma}[Transitivity of $<:$]\label{lem:sub-trans}
    If $A <: B$ and $B <: C$, then $A <: C$.
\end{lemma}


\begin{theorem}[Equivalence of the subtyping relations]\ \\
  $A <: B$ if and only if $A ≤ B$.
\end{theorem}

\clearpage
\pagebreak

\bibliographystyle{abbrvnat}
\bibliography{all}

\end{document}

% LocalWords:  dom subtyping Barendregt et al subformula sequent aa
% LocalWords:  BCD lclr refl glb topInCod
