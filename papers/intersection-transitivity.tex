\documentclass{article}
\usepackage{natbib}
\usepackage{amsthm}
\usepackage{amsmath}
\usepackage{amssymb}
\usepackage{mathabx}
\usepackage{stmaryrd}
\usepackage{semantic}
%\usepackage{fullpage}
\usepackage{fontspec,unicode-math}
\usepackage{hyperref}

\newtheorem{theorem}{Theorem}
\newtheorem{lemma}[theorem]{Lemma}
\newtheorem{corollary}[theorem]{Corollary}
\newtheorem{proposition}[theorem]{Proposition}
\newtheorem{constraint}[theorem]{Constraint}
\newtheorem{definition}[theorem]{Definition}
\newtheorem{example}[theorem]{Example}


\title{Transitivity of Subtyping for Intersection Types}
\author{Jeremy G. Siek}

\begin{document}
\maketitle

\newcommand{\TOP}{\ensuremath{\mathtt{U}}}
\newcommand{\dom}[1]{\mathsf{dom}(#1)}
\newcommand{\cod}[1]{\mathsf{cod}(#1)}
\newcommand{\topP}[1]{\mathsf{top}(#1)}
\newcommand{\topInCod}[1]{\mathsf{topInCod}(#1)}
\newcommand{\depth}[1]{\mathsf{depth}(#1)}
\newcommand{\size}[1]{\mathsf{size}(#1)}


\begin{abstract}
  The subtyping relation for intersection type systems traditionally
  employs a transitivity rule (Barendregt et al. 1983), which means
  that the subtyping judgment does not enjoy the subformula property.
  Laurent develops a sequent-style subtyping judgment, without
  transitivity, and proves transitivity via a sequence of six lemmas
  that culminate in cut-elimination (2018). This article presents a
  subtyping judgment, in regular style, that satisfies the subformula
  property, and presents a direct proof of transitivity. Borrowing
  from Laurent's system, the rule for function types is essentially
  the $\beta$-soundness property.  The main lemma required for the
  transitivity proof is one that has been used to prove the inversion
  principle for subtyping of function types. The choice of induction
  principle for the proof of transitivity is subtle: we use
  well-founded induction on the lexicographical ordering of the sum of
  the depths of the first and last type followed by the sum of the
  sizes of the middle and last type. The article concludes with a
  proof that the new subtyping judgment is equivalent to that of
  Barendregt, Coppo, and Dezani-Ciancaglini.
\end{abstract}

\section{Introduction}

Intersection types were invented by Coppo, Dezani-Ciancaglini, and
Salle, as a tool for studying normalization in the lambda
calculus~\citep{Coppo:1979aa}. By varying the subtyping rules and atom
types, researchers use intersection type systems to model many
different
calculi~\citep{Coppo:1980ab,Coppo:1981aa,Engeler:1981aa,Coppo:1984aa,Honsell:1992aa,Abramsky:1993fk,Plotkin:1993ab,Honsell:1999aa,Ishihara:2002aa,Rocca:2004aa,Dezani-Ciancaglini:2005aa,Alessi:2006aa}.
Perhaps the best-known of them is the BCD intersection type system of
\citet{Barendregt:1983aa}. For this article we focus on the BCD
system, following the presentation of \citet{Barendregt:2013aa}.  We
conjecture that our results apply to other intersection type systems
as well.

The BCD intersection type systems extends the simply-typed lambda
calculus with the addition of intersection types, written $A ∩ B$, a
top type $\TOP$, and an infinite collection of type
constants. Figure~\ref{fig:types} defines the grammar of types.

\begin{figure}[tbp]
  \[
  \begin{array}{lclr}
    α,\beta & ::= & \TOP \mid c_0 \mid c_1 \mid c_2 \mid \cdots & \text{atoms}\\
    A,B,C,D & ::= & α \mid A → B \mid A ∩ B & \text{types}
  \end{array}
  \]
  \caption{Intersection Types}
  \label{fig:types}
\end{figure}

The BCD intersection type system includes a subsumption rule which
states that a term $M$ in environment $\Gamma$ can be given type $B$
if it has type $A$ and $A$ is a subtype of $B$, written $A ≤ B$.
\[
\inference{\Gamma \vdash M : A & A ≤ B}
          {\Gamma \vdash M : B}
\]
Figure~\ref{fig:BCD-subtyping} reviews the BCD rules for subtyping.
Note that in the (trans) rule, the type $B$ that appears in the
premises does not appear in the conclusion. Thus, the BCD subtyping
judgment does not enjoy the subformula property.  For other systems,
it is straightforward to remove the (trans) rule, modify the other
rules, and then then prove transitivity~\citep{Muehlboeck:2018aa}.
Unfortunately, the $({→}{∩})$ rule of the BCD system significantly
complicates the situation.

\begin{figure}[tbp]
  \fbox{$A ≤ B$}
  \begin{gather*}
    \text{(refl)} \; \inference{}{A ≤ A} \quad
    \text{(trans)} \; \inference{A ≤ B & B ≤ C}{A ≤ C} \\[3ex]
    \text{(incl$_L$)} \; \inference{}{A ∩ B ≤ A} \quad
    \text{(incl$_R$)} \; \inference{}{A ∩ B ≤ B} \quad
    \text{(glb)} \; \inference{A ≤ C & A ≤ D}{A ≤ C ∩ D} \\[3ex]
    (→) \; \inference{C ≤ A & B ≤ D}{A → B ≤ C → D} \quad
    ({→}{∩}) \; \inference{}{(A → B) ∩ (A → C) ≤ A → (B ∩ C)} \\[3ex]
    (\TOP_{\mathrm{top}}) \; \inference{}{A ≤ \TOP} \quad
    (\TOP{→}) \; \inference{}{\TOP ≤ C → \TOP}
  \end{gather*}
  \caption{Subtyping of Barendregt, Coppo, and
    Dezani-Ciancaglini (BCD).}
  \label{fig:BCD-subtyping}
\end{figure}

The subformula property is a useful one. For example, the author is
using intersection types to create a denotational semantics for the
ISWIM language, which includes constants and primitive
operations~\citep{Landin:1966la,G.-D.-Plotkin:1975on,Felleisen:2009aa}.
It seems that doing so requires placing extra conditions on types and
it is much easier to do so when subtyping satisfies the subformula
property.

\citet{Laurent:2018aa} introduces the ISC sequent-style system,
written $\Gamma \vdash B$, where $\Gamma$ is a sequence of types
$A_1,\ldots,A_n$. The intuition is that $A_1,\ldots,A_n \vdash B$
corresponds to $A_1 ∩ \cdots ∩ A_n ≤ B$. The ISC system satisfies the
subformula property and is equivalent to the BCD system. To prove
this, Laurent establishes six lemmas that culminate in
cut-elimination, from which transitivity follows.

This article presents a more direct route to the subformula property
and transitivity. We present a subtyping relation $A <: B$ and
directly prove transitivity without using an auxiliary sequent-style
system. Nevertheless, the intuitions are based on those of
Laurent. The key to $A <: B$ is a rule for function types based on the
$\beta$-soundness property~\citep{Barendregt:2013aa}, just as in ISC.
The definitions and results in this article have been machine checked
in Agda.

We describe the new subtyping relation in
Section~\ref{sec:new-subtyping}, prove transitivity in
Section~\ref{sec:trans}, and prove its equivalence to BCD subtyping in
Section~\ref{sec:equiv}. We make some concluding remarks in
Section~\ref{sec:conclude}.
    
%$λ^{\mathrm{BCD}}_∩$

\section{A New Subtyping Judgment}
\label{sec:new-subtyping}

Our new subtyping judgment relies on several auxiliary notions that
help us avoid the use of ellipses, which we define in
Figure~\ref{fig:aux}. These include the $\dom{A}$ and $\cod{A}$
functions, the $\topP{A}$ and $\topInCod{A}$ predicates, and the
relations $A ∈ B$ and $A ⊆ B$.
%
The $\dom{A}$ and $\cod{A}$ functions return the domain or codomain if
$A$ is a function type, respectively. If $A$ is an intersection $A_1 ∩
A_2$, then $\dom{A}$ is the intersection of the domain of $A_1$ and
$A_2$.  If $A$ is an atom, $\dom{A}$ is undefined. Likewise for
$\cod{A}$. For example, if $A = (A_1 → B_1) ∩ \cdots ∩ (A_n → B_n)$,
then $\dom{A} = A_1 ∩ \cdots ∩ A_n$ and $\cod{A} = B_1 ∩ \cdots ∩
B_n$.  When $\dom{A}$ or $\cod{A}$ appears in lemma or theorem
statement, we implicitly assume that $A$ is a type such that $\dom{A}$
and $\cod{A}$ are defined.
%
The $\topP{A}$ predicate identifies types that are equivalent to
$\TOP$. The $\topInCod{A}$ predicate identifies types that have $\TOP$
in their codomain.
%
The relation $A ∈ B$ indicates whether $A$ is syntactically a part of $B$.
The relation $A ⊆ B$ holds when every part of $A$ is a part of $B$.
We say that $B$ \emph{contains} $A$ if $A ⊆ B$.

\begin{proposition}\label{prop:⊔⊆-inv}
 \item If $A ∩ B ⊆ C$, then $A ⊆ C$ and $B ⊆ C$. 
\end{proposition}


\begin{figure}[tbp]

  \fbox{$\dom{A}, \cod{A}$}
  \begin{align*}
  \dom{A → B} &= A \\
  \dom{A ∩ B} &= \dom{A} ∩ \dom {B} \\
  \\
  \cod{A → B} &= B \\
  \cod{A ∩ B} &= \cod{A} ∩ \cod {B}
  \end{align*}

  \fbox{$\topP{A}$}
  \begin{gather*}
    \inference{}{\topP{\TOP}}
    \quad
    \inference{\topP{B}}{\topP{A → B}}
    \quad
    \inference{\topP{A} & \topP{B}}{\topP{A ∩ B}}
  \end{gather*}

  \fbox{$A ∈ B$}
  \begin{gather*}
    \inference{}{α ∈ α}  \quad
    \inference{}{A → B ∈ A → B} \quad
    \inference{A ∈ B}{A ∈ B ∩ C} \quad
    \inference{A ∈ C}{A ∈ B ∩ C}
  \end{gather*}

  \fbox{$A ⊆ B$}
  \[
     A ⊆ B = ∀ C.\, C ∈ A \text{ implies } C ∈ B
  \]

  \fbox{$\mathsf{topInCod}(D)$}
  \[
  \mathsf{topInCod}(D) =
     \exists A B.\, A → B ∈ D \text{ and } \mathsf{top}(B)  
  \]

  \caption{Auxiliary Definitions}
  \label{fig:aux}
\end{figure}


\begin{figure}[tbp]
  \fbox{$A <: B$}
  \begin{gather*}
    \text{(refl$_α$)} \; \inference{}{α <: α} \\[3ex]
    \text{(lb$_L$)} \; \inference{A <: C}{A ∩ B <: C} \quad
    \text{(lb$_R$)} \; \inference{B <: C}{A ∩ B <: C} \quad
    \text{(glb)} \; \inference{A <: C & A <: D}{A <: C ∩ D} \\[3ex]
    (→') \; \inference{C <: \dom{B} & \cod{B} <: D }{A <: C → D}
    \begin{array}{l} \neg\, \mathsf{top}(D) \\
      \neg \, \mathsf{topInCod}(B) \\
      B ⊆ A \end{array}\\[3ex]
    (\TOP_{\mathrm{top}}) \; \inference{}{A <: \TOP} \quad
    (\TOP{→}') \; \inference{}{A <: C → D}\;\mathsf{top}(D)
  \end{gather*}
  \caption{The New Subtyping Judgment}
  \label{fig:new-subtyping}
\end{figure}


The new intersection subtyping judgment, $A <: B$, is defined in
Figure~\ref{fig:new-subtyping}. First, it does not include the (trans)
rule.  It also replaces the (refl) rule with reflexivity for atoms
(refl$_α$). The most important rule is the one for function types
$(→')$, which subsumes $(→)$ and $({→}{∩})$ in BCD subtyping.  The
$(→')$ rule essentially turns the $\beta$-soundness property into a
subtyping rule. The $(→')$ rule says that a type $A$ is a subtype of a
function type $C → D$ if a subset of $A$, call it $B$, has domain and
codomain that are larger and smaller than $C$ and $D$,
respectively. The use of a subset of $A$ enables this rule to absorb
uses of (incl$_L$) and (incl$_R$) on the left.  The side conditions
$\neg\;\topP{B}$ and $\neg\;\topInCod{D}$ are needed because of the
(\TOP{→}') rule, which in turn is needed to preserve types under
$\eta$-reduction.  In a system that does not involve $\eta$-reduction,
the (\TOP{→}') rule can be omitted, as well as those side
conditions. The rules (lb$_L$) and (lb$_R$) adapt (incl$_L$) and
(incl$_R$) to a system without transitivity, and have appeared many
times in the literature~\citep{Bakel:1995aa}.  The $(\TOP{→}')$ rule
generalizes the (\TOP{→}) rule, replacing the $\TOP$ on the left with
any type $A$, because for transitivity, any type is below $\TOP$. The
$(\TOP{→}')$ rule also replaces the $\TOP$ in the codomain on the
right with any type $D$ that is equivalent to $\TOP$.

Before moving on, we make note of some basic facts regarding the $<:$
relation and the $\topP{A}$ predicate.

\begin{proposition}[Basic Properties of $<:$]\ \label{prop:subtyping}
  \begin{enumerate}
  \item (reflexivity) $A <: A$ \label{prop:⊑-refl}
  \item If $A <: B ∩ C$, then $A <: B$ and $A <: C$. \label{prop:⊔⊑-inv}
  \item If $A <: B$ and $C ∈ B$, then $A <: C$.\label{prop:u∈v⊑w→u⊑w}
  \item If $A <: B$ and $C ⊆ B$, then $A <: C$.\label{prop:u⊆v⊑w→u⊑w}
  \end{enumerate}
\end{proposition}
\begin{proof}\ 
  \begin{enumerate}
  \item The proof of reflexivity is by induction on $A$. In the case
    $A = A_1 → A_2$, we proceed by cases on whether $\topP{A_2}$.
    If it is, deduce $A_1 → A_2 <: A_1 → A_2$ by rule $(\TOP{→}')$.
    Otherwise, apply rule $(→')$
  \item The proof is by induction on the derivation of $A <: B ∩ C$.
  \item The proof is by induction on $B$. In the case where $B = B_1
    ∩ B_2$, either $C ∈ B_1$ or $C ∈ B_2$, but in either case part
    \ref{prop:⊔⊑-inv} of this proposition fulfills the premise of
    the induction hypothesis, from which the conclusion follows.
  \item The proof  is by induction on $C$, using part \ref{prop:u∈v⊑w→u⊑w}
    of this proposition in the cases for atoms and function types.
  \end{enumerate}
\end{proof}

\begin{proposition}[Properties of $\topP{A}$]\ \label{prop:top}
  \begin{enumerate}
  \item If $\topP{A}$ then $\topP{\cod{A}}$.\label{prop:AllBot-cod}
  \item If $\topP{A}$ and $B ∈ A$, then $\topP{B}$.\label{prop:AllBot-∈}
  \item If $\topP{A}$ and $B ⊆ A$, then $\topP{B}$.\label{prop:AllBot-⊆}
  \item If $\topP{A}$ and $A <: B$, then $\topP{B}$.\label{prop:AllBot-⊑}
  \item If $\topP{A}$, then $B <: A$.\label{prop:AllBot-⊑-any}
  \end{enumerate}
\end{proposition}
\begin{proof}\
  \begin{enumerate}
  \item The proof is a straightforward induction on $A$.
  \item The proof is also a straightforward induction on $A$.
  \item The proof is by induction on $B$. The cases for atoms and
    function types are proved by part \ref{prop:AllBot-∈} of
    this proposition. In the case for
    $B = B_1 ∩ B_2$, from $B_1 ∩ B_2 <: A$, we have
    $B_1 <: A$ and $B_2 <: A$ (Proposition~\ref{prop:⊔⊆-inv}).
    Then by the induction hypotheses for $B_1$ and $B_2$ we have
    $\topP{B_1}$ and $\topP{B_2}$, from which we conclude
    that $\topP{B_1 ∩ B_2}$.
  \item The proof is by induction on the derivation of $A <: B$.
    All of the cases are straightforward except for rule $(→')$.
    In that case we have $B = B_1 → B_2$ and some $A'$ such that
    $A' ⊆ A$, $B_1 <: \dom{A'}$, $\cod{A'} <: B_2$, $\neg\,\topP{B_2}$,
    and $\neg\,\topInCod{A'}$. From the premise $\topP{A}$
    and part \ref{prop:AllBot-⊆} of this proposition, we
    have $\topP{A'}$. Then by part \ref{prop:AllBot-cod}
    we have $\topP{\cod{A'}}$. By the induction hypothesis
    for $\cod{A'} <: B$ we conclude that $\topP{B}$.
  \item The proof is a straightforward induction on $A$.
  \end{enumerate}
\end{proof}

Next we turn to the subtyping inversion principle for function types.
The idea is to generalize the rule $(→')$ with respect to the type on
the right, allowing any type that contains a function type.  The
premises of $(→')$ are somewhat complex, so we package most of them
into the following definition.

\begin{definition}[factors]
  We say $C → D$ \emph{factors} $A$
  if there exists some type $B$ such that
  $B ⊆ A$, $C <: \dom{B}$, $\cod{B} <: D$, and
  $\neg\,\mathsf{topInCod}(B)$.
\end{definition}

\begin{proposition}[Inversion Principle for Function Types]
  \label{prop:⊑-fun-inv}
  If $A <: C'$, $C → D ∈ C'$, and $\neg\,\mathsf{top}(D)$, then
  $C → D$ factors $A$.
\end{proposition}
\begin{proof}
  The proof is a straightforward induction on $A <: C'$.
\end{proof}

\section{Transitivity}
\label{sec:trans}

The proof of transitivity relies on the following lemma, which is
traditionally needed to prove the inversion principle for function
types. However, it was not needed for our system because the rule
$(→')$ is already quite close to the inversion principle.
The lemma states that if every function type $C → D$ in A
factors $B$, then $\dom{A} → \cod{A}$ also factors $B$.

\begin{lemma}\label{lem:sub-inv-trans}
  If
  \begin{itemize}
  \item for any $C$ $D$, if $C → D ∈ A$ and $\neg\,\mathsf{top}(D)$,
    then $C → D$ factors $B$, and
  \item $\neg\, \mathsf{topInCod}(A)$,
  \end{itemize}
  then $\dom{A} → \cod{A}$ factors $B$.
\end{lemma}
\begin{proof}
  The proof is by induction on $A$.
  \begin{itemize}
  \item Case $A$ is an atom. The statement is vacuously true.
  \item Case $A = A_1 → A_2$ is a function type. Then we conclude by applying
    the premise with $C$ and $D$ instantiated to $A_1$ and $A_2$ respectively.
  \item Case $A = A_1 ∩ A_2$.  By the induction hypothesis for $A_1$
    and for $A_2$, we have that $\dom{A_1} → \cod{A_1}$ factors $B$
    and so does $\dom{A_2} → \cod{A_2}$.  So there exists
    $B_1$ and $B_2$ such that $B_1 ⊆ B$, $\neg\,\topInCod{B_1}$,
    $\dom{A_1} <: \dom{B_1}$, $\cod{B_1} <: \cod{A_2}$ and similarly
    for $B_2$. We need to show that $\dom{A} → \cod{A}$ factors
    $B$. We choose the witness $B_1 ∩ B_2$.  Clearly we have $B_1 ∩
    B_2 ⊆ B$ and $\neg \topInCod{B_1 ∩ B_2}$.  Also, we have
    \[
    \dom{B_1} ∩ \dom{B_2} <: \dom{A_1} ∩ \dom{A_2}
    \]
    and
    \[
    \cod{A_1} ∩ \cod{A_2} <: \cod{B_1} ∩ \cod{B_2}
    \]
    Thus, we have that $\dom{B_1 ∩ B_2} <: \dom{A}$
    and $\cod{A} <: \cod{B_1 ∩ B_2}$, and this case is complete.
  \end{itemize}
\end{proof}

We now turn to the proof of transitivity, that if $A <: B$ and $B <:
C$, then $A <: C$. The proof is by well-founded induction on the
lexicographical ordering of the sum of the depths of $A$ and $C$
followed by the sum of the sizes of $B$ and $C$. To be precise, we
define this ordering as follows.
\[
\langle A, B, C \rangle \ll \langle A', B', C' \rangle
=
\begin{array}{l}
\depth{A} + \depth{C} < \depth{A'} + \depth{C'}
\text{ or } \\
(\depth{A} + \depth{C}
\leq
\depth{A'} + \depth{C'} \\
\text{ and }
\size{B} + \size{C}
<
\size{B'} + \size{C'})
\end{array}
\]
where $\size{A}$ is 
\begin{align*}
  \size{α} &= 0 \\
  \size{A → B} &= 1 + \size{A} + \size{B} \\
  \size{A ∩ B} &= 1 + \size{A} + \size{B}
\end{align*}
and $\depth{A}$ is 
\begin{align*}
  \depth{α} &= 0 \\
  \depth{A → B} &= 1 + \max(\depth{A}, \depth{B}) \\
  \depth{A ∩ B} &= \max(\depth{A}, \depth{B})
\end{align*}

\begin{theorem}[Transitivity of $<:$]\label{thm:⊑-trans}
    If $A <: B$ and $B <: C$, then $A <: C$.
\end{theorem}
\begin{proof}
  The proof is by well-founded induction on the relation $\ll$.
  We proceed by cases on the last rule applied in the
  derivation of $B <: C$.
  \begin{description}
  \item[Case (refl$_α$)] We have $B = C = α$.  From the premise $A <:
    B$ we immediately conclude that $A <: α$.
  \item[Case (lb$_L$)] So $B = B_1 ∩ B_2$, $B_1 <: C$, and $A <: B_1 ∩
    B_2$.  We have $A <: B_1$ (Proposition~\ref{prop:subtyping} part
    \ref{prop:⊔⊑-inv}), so we conclude that $A <: C$ by the induction
    hypothesis, noting that $\langle A, B_1, C \rangle \ll \langle A,
    B, C \rangle$ because $\size{B_1} < \size{B}$.
  \item[Case (lb$_R$)] So $B = B_1 ∩ B_2$, $B_2 <: C$, and $A <: B_1 ∩
    B_2$.  We have $A <: B_2$ (Proposition~\ref{prop:subtyping} part
    \ref{prop:⊔⊑-inv}), so we conclude that $A <: C$ by the induction
    hypothesis, noting that $\langle A, B_2, C \rangle \ll \langle A,
    B, C \rangle$ because $\depth{B_2} \leq \depth{B}$ and $\size{B_2}
    < \size{B}$.
  \item[Case (glb)] We have $C = C_1 ∩ C_2$, $B <: C_1$, and $B <:
    C_2$.  By the induction hypothesis, we have $A <: C_1$ and $A <:
    C_2$, noting that $\langle A, B, C_1 \rangle \ll \langle A, B, C
    \rangle$ $\langle A, B, C_2 \rangle \ll \langle A, B, C\rangle$
    because $\depth{C_1} \leq \depth{C}$, $\depth{C_2} \leq
    \depth{C}$, $\size{C_1} < C$, and $\size{C_2} < C$.
    We conclude $A <: C_1 ∩ C_2$ by rule (glb).
  \item[Case $(→')$] So $C = C_1 → C_2$, $\neg \topP{C_2}$, and there
    exists $B'$ such that $C_1 <: \dom{B'}$, $\cod{B'} <: C_2$, $B' ⊆
    B$, and $\neg \topInCod{B'}$. From $A <: B$ and $B' ⊆ B$, we have
    $A <: B'$ (Proposition~\ref{prop:subtyping} part
    \ref{prop:u⊆v⊑w→u⊑w}). Thus, for any $B_1 → B_2 ∈ B'$, $B_1 → B_2$
    factors $A$ (Proposition~\ref{prop:⊑-fun-inv}). We have
    satisfied the premises of Lemma~\ref{lem:sub-inv-trans},
    so $\dom{B'} → \cod{B'}$ factors $A$. That means there exists
    $A'$ such that $A' ⊆ A$, $\neg\,\topInCod{A'}$, $\dom{B'} <: \dom{A'}$,
    and $\cod{A'} <: \cod{B'}$. Then by the induction hypothesis,
    we have
    \[
    C_1 <: \dom{A'} \quad\text{and}\quad \cod{A'} <: C_2
    \]
    noting that
    $\langle C_1, \dom{B'}, \dom{A'} \rangle \ll \langle A, B, C \rangle$
    because $\depth{C_1} + \depth{\dom{A'}} < \depth{A} + \depth{C}$
    and
    $\langle \cod{A'}, \cod{B'}, C_2 \rangle \ll \langle A, B, C \rangle$
    because $\depth{\cod{A'}} + \depth{C_2} < \depth{A} + \depth{C}$.
    We conclude that $A <: C_1 → C_2$ by rule $(→')$ witnessed by $A'$.
  \item[Case $(\TOP_{\mathrm{top}})$] We have $C = \TOP$ and
    conclude $A <: \TOP$ by rule $(\TOP_{\mathrm{top}})$.
  \item[Case $(\TOP{→}')$] We have $C = C_1 → C_2$ and $\topP{C_2}$.
    We conclude $A <: C_1 → C_2$ by rule $(\TOP{→}')$.
  \end{description}
\end{proof}


\section{Equivalence with BCD Subtyping}
\label{sec:equiv}

Having proved (trans), we next prove $(→)$ and $({→}{∩})$
and then show that $A <: B$ is equivalent to $A ≤ B$.

\begin{lemma}[$→$]\label{lem:⊑-fun′}
  If $C <: A$ and $B <: D$, then
  $A → B <: C → D$.
\end{lemma}
\begin{proof}
  Consider whether $\topP{D}$ or not.
  \begin{description}
  \item[Case $\topP{D}$] We conclude $A → B <: C → D$
    by rule $(\TOP{→}')$.
  \item[Case $\neg \,\topP{D}$]
    Consider whether $\topP{B}$ or not.
    \begin{description}
    \item[Case $\topP{B}$] So $\topP{D}$ (Prop. \ref{prop:top}
      part \ref{prop:AllBot-⊑}), but that is a contradiction.
    \item[Case $\neg\,\topP{B}$]
      We conclude that $A → B <: C → D$ by rule $(→')$.
    \end{description}
  \end{description}
\end{proof}

\begin{lemma}[${→}{∩}$]\label{lem:⊑-dist}
  $(A → B) ∩ (A → C) <: A → (B ∩ C)$
\end{lemma}
\begin{proof}
   We consider the cases for whether $\topP{B}$ or $\topP{C}$.
   \begin{description}
   \item[Case $\topP{B}$ and $\topP{C}$]
     Then $\topP{B ∩ C}$ and we conclude that
     $(A → B) ∩ (A → C) <: A → (B ∩ C)$
     by rule $(\TOP{→}')$.
   \item[Case $\topP{B}$ and $\neg\,\topP{C}$]
     We conclude that 
     $(A → B) ∩ (A → C) <: A → (B ∩ C)$
     by rule $(→')$, choosing the witness $A → C$
     and noting that $C <: B$ 
     by way of Proposition~\ref{prop:top}
     part \ref{prop:AllBot-⊑-any}
     and $C <: C$ by Proposition~\ref{prop:subtyping}
     part \ref{prop:⊑-refl}.
   \item[Case $\neg\,\topP{B}$ and $\topP{C}$]
     We conclude that 
     $(A → B) ∩ (A → C) <: A → (B ∩ C)$
     by rule $(→')$, this time with witness $A → B$
     and noting that
     $B <: B$ by Proposition~\ref{prop:subtyping}
     part \ref{prop:⊑-refl}
     and $B <: C$ 
     by way of Proposition~\ref{prop:top}
     part \ref{prop:AllBot-⊑-any}.
   \item[Case $\neg,\topP{B}$ and $\neg\,\topP{C}$]
     Again we apply rule $(→')$, but with witness
     $(A → B) ∩ (A → C)$.
   \end{description}
\end{proof}

\noindent We require one more lemma.

\begin{lemma}
  \label{lem:dv↦cv<:v}
  $A ≤ \dom{A} → \cod{A}$.
\end{lemma}
\begin{proof}
  The proof is by induction on $A$.
\end{proof}

\noindent Now for the proof of equivalence

\begin{theorem}[Equivalence of the subtyping relations]\ \\
  $A <: B$ if and only if $A ≤ B$.
\end{theorem}
\begin{proof}
  We prove each direction of the if-and-only-if separately.
  \begin{description}
  \item[$A <: B$ implies $A ≤ B$]
    We proceed by induction on the derivation of $A <: B$.
    \begin{description}
    \item[Case (refl$_α$)] We conclude $α ≤ α$ by (refl).      
    \item[Case (lb$_L$)] By the induction hypothesis we have $A ≤ C$.
      By (incl$_L$) we have $A ∩ B ≤ A$. We conclude
      that $A ∩ B ≤ C$ by (trans).
    \item[Case (lb$_R$)] By the induction hypothesis we have $B ≤ C$.
      By (incl$_R$) we have $A ∩ B ≤ B$. We conclude
      that $A ∩ B ≤ C$ by (trans).
    \item[Case (glb)] By the induction hypothesis we have
      $A ≤ C$ and $A ≤ D$, so we conclude that
      $A ≤ C ∩ D$ by (glb).
    \item[Case $(→')$] By the induction hypothesis we have
      $C ≤ \dom{B}$ and also $\cod{B} ≤ D$.
      From $B ⊆ A$ we have $A ≤ B$.      
      Then by Lemma~\ref{lem:dv↦cv<:v} we have
      $B ≤ \dom{B} → \cod{B}$.
      Also, we have $\dom{B} → \cod{B} ≤ C → D$ by rule $(→)$.
      We conclude that $A ≤ C → D$ by chaining the three prior
      facts using (trans).
    \item[Case $(\TOP_{\mathrm{top}})$]
      We conclude that $A ≤ \TOP$ by $(\TOP_{\mathrm{top}})$.
    \item[Case $(\TOP{→}')$] We have $A ≤ \TOP$ and $\TOP <: C →
      \TOP$.  Also, $C → \TOP ≤ C → D$ because $\TOP <: D$ follows
      from $\topP{D}$.  Thus, applying (trans) we conclude $A ≤ C →
      D$.
    \end{description}
    
  \item[$A ≤ B$ implies $A <: B$]
    We proceed by induction on the derivation of $A ≤ B$.
    \begin{description}
    \item[Case (refl)] We conclude $A <: A$ by Prop.~\ref{prop:subtyping}
      part \ref{prop:⊑-refl}.
    \item[Case (trans)]
      By the induction hypothesis, we have $A <: B$ and $B <: C$.
      We conclude that $A <: C$ by Theorem~\ref{thm:⊑-trans}.
    \item[Case (incl$_L$)] We have $A <: A$
      (Prop.~\ref{prop:subtyping} part \ref{prop:⊑-refl}),
      and therefore $A ∩ B <: A$ by rule (lb$_L$).
    \item[Case (incl$_R$)] We have $B <: B$
      (Prop.~\ref{prop:subtyping} part \ref{prop:⊑-refl}),
      and therefore $A ∩ B <: B$ by rule (lb$_R$).
    \item[Case (glb)] By the induction hypothesis, we have
      $A <: C$ and $A <: D$, so we conclude that $A <: C ∩ D$ by (glb).
    \item[Case $(→)$] By the induction hypothesis, we have $C <: A$
      and $B <: D$. We conclude that $A → B <: C → D$
      by Lemma~\ref{lem:⊑-fun′}.
    \item[Case $({→}{∩})$] We conclude that
      $(A → B) ∩ (A → C) <: A → (B ∩ C)$
      by Lemma~\ref{lem:⊑-dist}.
    \item[Case $(\TOP_{\mathrm{top}})$]
      We conclude that $A <: \TOP$ by rule $(\TOP_{\mathrm{top}})$.
    \item[Case $(\TOP{→})$]
      We have $\topP{\TOP}$, so $\TOP <: C → \TOP$ by rule $(\TOP{→}')$.
    \end{description}
    
  \end{description}
\end{proof}

\section{Conclusion}
\label{sec:conclude}

In this article we present a new subtyping relation $A <: B$ for
intersection types that enjoys the subformula property.  None of the
rules of the new subtyping relation are particularly novel, but the
fact that we can prove transitivity directly from them is!  We prove
that the new relation is equivalent to the subtyping relation $A ≤ B$
of Barendregt, Coppo, and Dezani-Ciancaglini.

\pagebreak

\bibliographystyle{abbrvnat}
\bibliography{all}

\end{document}

% LocalWords:  dom subtyping Barendregt et al subformula sequent aa
% LocalWords:  BCD lclr refl glb topInCod Coppo Dezani Ciancaglini fk
% LocalWords:  Salle Engeler Honsell Abramsky Plotkin Rocca Alessi
% LocalWords:  subsumption subtype denotational ISWIM Landin Ishihara
% LocalWords:  Felleisen Muehlboeck ISC Agda codomain reflexivity
% LocalWords:  Bakel
