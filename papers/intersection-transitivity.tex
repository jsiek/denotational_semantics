\documentclass{article}
\usepackage{natbib}
\usepackage{amsthm}
\usepackage{amsmath}
\usepackage{amssymb}
\usepackage{mathabx}
\usepackage{stmaryrd}
\usepackage{semantic}
%\usepackage{fullpage}
\usepackage{fontspec,unicode-math}

\newtheorem{theorem}{Theorem}
\newtheorem{lemma}[theorem]{Lemma}
\newtheorem{corollary}[theorem]{Corollary}
\newtheorem{proposition}[theorem]{Proposition}
\newtheorem{constraint}[theorem]{Constraint}
\newtheorem{definition}[theorem]{Definition}
\newtheorem{example}[theorem]{Example}


\title{Transitivity of Subtyping for Intersection Types}
\author{Jeremy G. Siek}

\begin{document}
\maketitle

\newcommand{\TOP}{\ensuremath{\mathtt{U}}}
\newcommand{\dom}[1]{\mathrm{dom}(#1)}
\newcommand{\cod}[1]{\mathrm{cod}(#1)}
\newcommand{\topP}[1]{\mathrm{top}(#1)}
\newcommand{\topInCod}[1]{\mathrm{topInCod}(#1)}


\begin{abstract}
  The subtyping relation for intersection type systems traditionally
  employs a transitivity rule (Barendregt et al. 1983), which means
  that the subtyping judgment does not enjoy the subformula property.
  Laurent develops a sequent-style subtyping judgment, without
  transitivity, and proves transitivity via a sequence of six lemmas
  that culminate in cut-elimination (2018). This article presents a
  subtyping judgment, in regular style, that satisfies the subformula
  property, and gives a direct proof of transitivity. Borrowing from
  Laurent's system, the rule for function types is essentially the
  $\beta$-soundness property.  The main lemma required for the
  transitivity proof is one that has been used to prove the inversion
  principle for subtyping of function types (Barendregt et
  al. 2013). The choice of induction principle for the proof of
  transitivity requires some care: we use strong induction on the
  lexicographical ordering of the sum of the depths of the first and
  last type followed by the sum of the sizes of the first and second
  type.
\end{abstract}

\section{Introduction}

Intersection types were invented by Coppo, Dezani-Ciancaglini, and
Salle, as a tool for studying normalization in the lambda
calculus~\citep{Coppo:1979aa}. By varying the subtyping rules and atom
types, researchers have used intersection type systems to model many
different
calculi~\citep{Coppo:1980ab,Coppo:1981aa,Engeler:1981aa,Coppo:1984aa,Honsell:1992aa,Abramsky:1993fk,Plotkin:1993ab,Honsell:1999aa,Ishihara:2002aa,Rocca:2004aa,Dezani-Ciancaglini:2005aa,Alessi:2006aa}.
Perhaps the best-known of them is the BCD intersection type system of
\citet{Barendregt:1983aa}. For this article we focus on the BCD
system, following the presentation of \citet{Barendregt:2013aa}.  We
conjecture that our results apply to other intersection type systems.

The BCD intersection type systems extends the simply-typed lambda
calculus with the addition of intersection types, written $A ∩ B$, a
top type $\TOP$, and an infinite collection of type
constants. Figure~\ref{fig:types} defines the grammar of types.

\begin{figure}[tbp]
  \[
  \begin{array}{lclr}
    \alpha,\beta & ::= & \TOP \mid c_0 \mid c_1 \mid c_2 \mid \cdots & \text{atoms}\\
    A,B,C,D & ::= & \alpha \mid A → B \mid A ∩ B & \text{types}
  \end{array}
  \]
  \caption{Intersection Types}
  \label{fig:types}
\end{figure}

The BCD intersection type system includes a subsumption rule which
states that a term $M$ in environment $\Gamma$ can be given type $B$
if it has type $A$ and $A$ is a subtype of $B$, written $A ≤ B$.
\[
\inference{\Gamma \vdash M : A & A ≤ B}
          {\Gamma \vdash M : B}
\]
Figure~\ref{fig:BCD-subtyping} reviews the BCD rules for subtyping.
Note that in the (trans) rule, the type $B$ that appears in the
premises does not appear in the conclusion. Thus, the BCD subtyping
judgment does not enjoy the subformula property.  For other systems,
it is straightforward to remove the (trans) rule, modify the other
rules, and then then prove transitivity~\citep{Muehlboeck:2018aa}.
Unfortunately, the ${→}{∩}$ rule significantly complicates the
situation for the BCD system.

\begin{figure}[tbp]
  \fbox{$A ≤ B$}
  \begin{gather*}
    \text{(refl)} \; \inference{}{A ≤ A} \quad
    \text{(trans)} \; \inference{A ≤ B & B ≤ C}{A ≤ C} \\[3ex]
    \text{(incl$_L$)} \; \inference{}{A ∩ B ≤ A} \quad
    \text{(incl$_R$)} \; \inference{}{A ∩ B ≤ B} \quad
    \text{(glb)} \; \inference{C ≤ A & C ≤ B}{C ≤ A ∩ B} \\[3ex]
    (→) \; \inference{A' ≤ A & B ≤ B'}{A → B ≤ A' → B'} \quad
    ({→}{∩}) \; \inference{}{(A → B) ∩ (A → C) ≤ A → (B ∩ C)} \\[3ex]
    (\TOP_{\mathrm{top}}) \; \inference{}{A ≤ \TOP} \quad
    (\TOP{→}) \; \inference{}{\TOP ≤ A → \TOP}
  \end{gather*}
  \caption{Subtyping of Barendregt, Coppo, and
    Dezani-Ciancaglini (BCD).}
  \label{fig:BCD-subtyping}
\end{figure}

The subformula property is a useful one. For example, the author is
using intersection types to create a denotational semantics for the
ISWIM language, which includes constants and primitive
operations~\citep{Landin:1966la,G.-D.-Plotkin:1975on,Felleisen:2009aa}.
It seems that doing so requires placing extra conditions on types and
that is much easier when subtyping satisfies the subformula property.

\citet{Laurent:2018aa} introduces the ISC sequent-style system,
written $\Gamma \vdash B$, where $\Gamma$ is a sequence of types
$A_1,\ldots,A_n$. The intuition is that $A_1,\ldots,A_n \vdash B$
corresponds to $A_1 ∩ \cdots ∩ A_n ≤ B$. The ISC system satisfies the
subformula property and is equivalent to the BCD system. To prove
this, Laurent establishes six lemmas that culminate in
cut-elimination, from which transitivity follows dirctly.

This article presents a more direct route to the subformula property
and transitivity. We present a subtyping relation $A <: B$ and
directly prove transitivity, without using an auxilliary sequent-style
system. Nevertheless, the intuitions are based on those of
Laurent. The key to $A <: B$ is a rule for function types based on the
$\beta$-soundness property~\citep{Barendregt:2013aa}, just as in ISC.
The definitions and results in this article have been machine checked
in Agda.


ROAD MAP


%$λ^{\mathrm{BCD}}_∩$

\section{Intersection Subtyping, with Subformula}

Our new subtyping judgment relies on several auxilliary notions that
help us avoid the use of ellipses, which we define in
Figure~\ref{fig:aux}. These include the $\dom{A}$ and $\cod{A}$
functions, the $\topP{A}$ and $\topInCod{A}$ predicates, and the
relations $A ∈ B$ and $A ⊆ B$.
%
The $\dom{A}$ and $\cod{A}$ functions return the domain or codomain if
$A$ is a function type, respectively. If $A$ is an intersection $A_1 ∩
A_2$, then $\dom{A}$ is the intersection of the domain of $A_1$ and
$A_2$.  If $A$ is an atom, $\dom{A}$ is undefined. Likewise for
$\cod{A}$. For example, if $A = (A_1 → B_1) ∩ \cdots ∩ (A_n → B_n)$,
then $\dom{A} = A_1 ∩ \cdots ∩ A_n$ and $\cod{A} = B_1 ∩ \cdots ∩ B_n$.
%
The $\topP{A}$ predicate identifies types that are equivalent to
$\TOP$. The $\topInCod{A}$ predicate identifies types that have $\TOP$
in their codomain.
%
The relation $A ∈ B$ indicates whether $A$ is syntactically a part of $B$.
The relation $A ⊆ B$ holds when every part of $A$ is a part of $B$.
We say that $B$ \emph{contains} $A$ if $A ⊆ B$.

\begin{proposition}
\item If $A ∩ B ⊆ C$, then $A ⊆ C$ and $B ⊆ C$. \label{prop:⊔⊆-inv}
\end{proposition}


\begin{figure}[tbp]

  \fbox{$\dom{A}, \cod{A}$}
  \begin{align*}
  \dom{A → B} &= A \\
  \dom{A ∩ B} &= \dom{A} ∩ \dom {B} \\
  \\
  \cod{A → B} &= B \\
  \cod{A ∩ B} &= \cod{A} ∩ \cod {B}
  \end{align*}

  \fbox{$\topP{A}$}
  \begin{gather*}
    \inference{}{\topP{\TOP}}
    \quad
    \inference{\topP{B}}{\topP{A → B}}
    \quad
    \inference{\topP{A} & \topP{B}}{\topP{A ∩ B}}
  \end{gather*}

  \fbox{$A ∈ B$}
  \begin{gather*}
    \inference{}{\alpha ∈ \alpha}  \quad
    \inference{}{A → B ∈ A → B} \quad
    \inference{A ∈ B}{A ∈ B ∩ C} \quad
    \inference{A ∈ C}{A ∈ B ∩ C}
  \end{gather*}

  \fbox{$A ⊆ B$}
  \[
     A ⊆ B = ∀ C.\, C ∈ A \text{ implies } C ∈ B
  \]

  \fbox{$\mathsf{topInCod}(D)$}
  \[
  \mathsf{topInCod}(D) =
     \exists A B.\, A → B ∈ D \text{ and } \mathsf{top}(B)  
  \]

  \caption{Auxiliary Definitions}
  \label{fig:aux}
\end{figure}


\begin{figure}[tbp]
  \fbox{$A <: B$}
  \begin{gather*}
    \text{(refl$_α$)} \; \inference{}{α <: α} \\[3ex]
    \text{(lb$_L$)} \; \inference{A <: C}{A ∩ B <: C} \quad
    \text{(lb$_R$)} \; \inference{B <: C}{A ∩ B <: C} \quad
    \text{(glb)} \; \inference{C ≤ A & C ≤ B}{C ≤ A ∩ B} \\[3ex]
    (→') \; \inference{C <: \dom{B} & \cod{B} <: D }{A <: C → D}
    \begin{array}{l} \neg\, \mathsf{top}(D) \\
      \neg \, \mathsf{topInCod}(B) \\
      B ⊆ A \end{array}\\[3ex]
    (\TOP_{\mathrm{top}}) \; \inference{}{A <: \TOP} \quad
    (\TOP{→}') \; \inference{}{C <: A → B}\;\mathsf{top}(B)
  \end{gather*}
  \caption{The New Subtyping Judgment}
  \label{fig:new-subtyping}
\end{figure}


The new intersection subtyping judgment, $A <: B$, is defined in
Figure~\ref{fig:new-subtyping}. Of course, it omits the (trans) rule.
We also omit the (refl) rule, replacing it with reflexivity for atoms
(refl$_\alpha$). The most important rule is the one for function types
$(→')$, which subsumes the rules $(→)$ and $({→}{∩})$ in BCD
subtyping. It essentially turns the $\beta$-soundness property into a
subtyping rule. It says that a type $A$ is a subtype of a function
type $C → D$ if $A$ contains a type $B$ whose domain and codomain are
larger and smaller than $C$ and $D$, respectively. The use of $B$
instead of $A$ in the premise, with the condition $B ⊆ A$, allows this
rule to absorb uses of (incl$_L$) and (incl$_R$) on the left.  The
side conditions $\neg\;\topP{B}$ and $\neg\;\topInCod{D}$ are needed
because of the (\TOP{→}') rule, which in turn is needed to preserve
types under $\eta$-reduction.  In a system that does not involve
$\eta$-reduction, the (\TOP{→}') rule can be ommitted, as well as
those side conditions. The rules (lb$_L$) and (lb$_R$) adapt
(incl$_L$) and (incl$_R$) to a system without transitivity, and have
appeared many times in the literature~\citep{Bakel:1995aa}.


\begin{proposition}\ 
  \begin{enumerate}
  \item (reflexivity) $A <: A$ \label{prop:refl}
  \item If $A <: B ∩ C$, then $A <: B$ and $A <: C$. \label{prop:⊔⊑-inv}
  \end{enumerate}
\end{proposition}
\begin{proof}
  The proof of reflexivity is by induction on $A$. In the case
  $A = A_1 → A_2$, we proceed by cases on whether $\topP{A_2}$.
  If it is, deduce $A_1 → A_2 <: A_1 → A_2$ by rule $(\TOP{→}')$.
  Otherwise, we use rule $(→')$
\end{proof}

\begin{proposition}\label{prop:u⊆v⊑w→u⊑w}
 If $A <: B$ and $C ⊆ B$, then $A <: C$.
\end{proposition}

\clearpage
\pagebreak
\section{Transitivity}


\begin{definition}[factors]
  We say $A → B$ \emph{factors} $C$
  if there exists some type $C'$ such that
  $C' ⊆ C$, $\neg\,\mathsf{topInCod}(C')$, $A <: \dom{C'}$, and $\cod{C'} <: B$.
\end{definition}

\begin{lemma}\label{lem:⊑-fun-inv}
  If $\neg\,\mathsf{top}(B)$,
  $D <: C$, and
  $A → B ∈ C$, then
  $A → B$ factors $D$.
\end{lemma}


\begin{lemma}\label{lem:sub-inv-trans}
  If
  \begin{itemize}
  \item $\neg\, \mathsf{topInCod}(D)$,
  \item for any $A,B$, if $\neg\,\mathsf{top}(B)$ and $A → B ∈ D$,
    then $A → B$ factors $C$,
  \end{itemize}
  then $\dom{D} → \cod{D}$ factors $C$.
\end{lemma}

\begin{align*}
  \mathsf{size}(\alpha) &= 0 \\
  \mathsf{size}(A → B) &= 1 + \mathsf{size}(A) + \mathsf{size}(B) \\
  \mathsf{size}(A ∩ B) &= 1 + \mathsf{size}(A) + \mathsf{size}(B)
\end{align*}

\begin{align*}
  \mathsf{depth}(\alpha) &= 0 \\
  \mathsf{depth}(A → B) &= 1 + \max(\mathsf{depth}(A), \mathsf{depth}(B)) \\
  \mathsf{depth}(A ∩ B) &= \max(\mathsf{depth}(A), \mathsf{depth}(B))
\end{align*}

\begin{theorem}[Transitivity of $<:$]\label{lem:sub-trans}
    If $A <: B$ and $B <: C$, then $A <: C$.
\end{theorem}

\section{Equivalence of BCD Subtyping and the new Subtyping}


\begin{theorem}[Equivalence of the subtyping relations]\ \\
  $A <: B$ if and only if $A ≤ B$.
\end{theorem}

\clearpage
\pagebreak

\bibliographystyle{abbrvnat}
\bibliography{all}

\end{document}

% LocalWords:  dom subtyping Barendregt et al subformula sequent aa
% LocalWords:  BCD lclr refl glb topInCod Coppo Dezani Ciancaglini fk
% LocalWords:  Salle Engeler Honsell Abramsky Plotkin Rocca Alessi
% LocalWords:  subsumption subtype denotational ISWIM Landin
% LocalWords:  Felleisen
