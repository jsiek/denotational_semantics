\documentclass{article}
%\usepackage{cmbright}
\usepackage{natbib}
\usepackage{amsthm}
\usepackage{amsmath}
\usepackage{amssymb}
%\usepackage{mathabx}
\usepackage{stmaryrd}
\usepackage{semantic}
%\usepackage{fullpage}
%\usepackage{fontspec,unicode-math}
\usepackage{hyperref}

\newtheorem{theorem}{Theorem}
\newtheorem{lemma}[theorem]{Lemma}
\newtheorem{corollary}[theorem]{Corollary}
\newtheorem{proposition}[theorem]{Proposition}
\newtheorem{constraint}[theorem]{Constraint}
\newtheorem{definition}[theorem]{Definition}
\newtheorem{example}[theorem]{Example}


\title{Transitivity of Subtyping for Intersection Types}
\author{Jeremy G. Siek}

\begin{document}
\maketitle

\newcommand{\TOP}{\ensuremath{\mathtt{U}}}
\newcommand{\dom}[1]{\mathsf{dom}(#1)}
\newcommand{\cod}[1]{\mathsf{cod}(#1)}
\newcommand{\topP}[1]{\mathsf{top}(#1)}
\newcommand{\topInCod}[1]{\mathsf{topInCod}(#1)}
\newcommand{\depth}[1]{\mathsf{depth}(#1)}
\newcommand{\size}[1]{\mathsf{size}(#1)}


\begin{abstract}
  The subtyping rules for intersection type systems traditionally
  employ a transitivity rule (Barendregt et al. 1983), which means
  that subtyping does not enjoy the subformula property.  Laurent
  develops a sequent-style subtyping system, without transitivity, and
  proves transitivity via a sequence of six lemmas that culminate in
  cut-elimination (2018). This article develops a subtyping system in
  regular style that omits transitivity and provides a direct proof of
  transitivity, significantly reducing the length of the proof. The
  new system satisfies something like the subformula property: every
  type occuring in the derivation of $A <: B$ is a subformula of $A$
  or $B$, or an intersection of such subformulas.  Borrowing from
  Laurent's system, the rule for function types is essentially the
  $\beta$-soundness property.  The main lemma required for the
  transitivity proof is one that has been used to prove the inversion
  principle for subtyping of function types. The article concludes
  with a proof that the new subtyping system is equivalent to that of
  Barendregt, Coppo, and Dezani-Ciancaglini.
\end{abstract}

\section{Introduction}

Intersection types were invented by Coppo, Dezani-Ciancaglini, and
Salle as a tool for studying normalization in the lambda
calculus~\citep{Coppo:1979aa}. Subsequently intersection types have
been used for at least three purposes: to enhance the expressiveness
of type systems
~\citep{Reynolds:1988aa,Pierce:1991aa,Castagna:2014aa,Chaudhuri:2014aa,Oliveira:2016aa,Muehlboeck:2018aa,Bi:2019aa,Dunfield:2019aa,Microsoft:TypeScript2020aa,Dotty:2020aa},
to increase the precision of static
analyses~\citep{Turbak:1997aa,Palsberg:1998aa,Mossin:2003aa,Simoes:2007aa},
and to express filter models for the denotational semantics of a
variety of lambda calculi
~\citep{Coppo:1980ab,Coppo:1981aa,Engeler:1981aa,Coppo:1984aa,Honsell:1992aa,Abramsky:1993fk,Plotkin:1993ab,Honsell:1999aa,Ishihara:2002aa,Rocca:2004aa,Dezani-Ciancaglini:2005aa,Alessi:2006aa}.

The work in this 

A filter model

\[
   \llbracket M \rrbracket = \{ A \mid \emptyset \vdash M : A \}
\]





Perhaps the best-known  them is the BCD intersection type system of
\citet{Barendregt:1983aa}. For this article we focus on the BCD
system, following the presentation of \citet{Barendregt:2013aa}.  We
conjecture that our results apply to other intersection type systems
that include the $({\to}{\cap})$ rule for distributing intersection
and function types.

The BCD intersection type systems extends the simply-typed lambda
calculus with the addition of intersection types, written $A \cap B$, a
top type $\TOP$, and an infinite collection of type
constants. Figure~\ref{fig:types} defines the grammar of types.

\begin{figure}[tbp]
  \[
  \begin{array}{lclr}
    \alpha,\beta & ::= & \TOP \mid c_0 \mid c_1 \mid c_2 \mid \cdots & \text{atoms}\\
    A,B,C,D & ::= & \alpha \mid A \to B \mid A \cap B & \text{types}
  \end{array}
  \]
  \caption{Intersection Types}
  \label{fig:types}
\end{figure}

The BCD intersection type system includes a subsumption rule which
states that a term $M$ in environment $\Gamma$ can be given type $B$
if it has type $A$ and $A$ is a subtype of $B$, written $A \leq B$.
\[
\inference{\Gamma \vdash M : A & A \leq B}
          {\Gamma \vdash M : B}
\]
Figure~\ref{fig:BCD-subtyping} reviews the BCD rules for subtyping.
Note that in the (trans) rule, the type $B$ appears in the premises
but not in the conclusion. Thus, the BCD subtyping judgment does not
enjoy the subformula property.  For many other subtyping systems, it
is straightforward to remove the (trans) rule, modify the other rules,
and then then prove transitivity.  Unfortunately, the $({\to}{\cap})$
rule of the BCD system significantly complicates the situation.

\begin{figure}[tbp]
  \fbox{$A \leq B$}
  \begin{gather*}
    \text{(refl)} \; \inference{}{A \leq A} \quad
    \text{(trans)} \; \inference{A \leq B & B \leq C}{A \leq C} \\[3ex]
    \text{(incl$_L$)} \; \inference{}{A \cap B \leq A} \quad
    \text{(incl$_R$)} \; \inference{}{A \cap B \leq B} \quad
    \text{(glb)} \; \inference{A \leq C & A \leq D}{A \leq C \cap D} \\[3ex]
    (\to) \; \inference{C \leq A & B \leq D}{A \to B \leq C \to D} \quad
    ({\to}{\cap}) \; \inference{}{(A \to B) \cap (A \to C) \leq A \to (B \cap C)} \\[3ex]
    (\TOP_{\mathrm{top}}) \; \inference{}{A \leq \TOP} \quad
    (\TOP{\to}) \; \inference{}{\TOP \leq C \to \TOP}
  \end{gather*}
  \caption{Subtyping of Barendregt, Coppo, and
    Dezani-Ciancaglini (BCD).}
  \label{fig:BCD-subtyping}
\end{figure}

The subformula property is a useful one. For example, the author uses
intersection types to create denotational semantics (a filter model)
for the ISWIM language, which includes constants and primitive
operations~\citep{Landin:1966la,G.-D.-Plotkin:1975on,Felleisen:2009aa}.
This requires placing extra conditions on types and it is much easier
to do so when subtyping satisfies the subformula property.

There has been a recent flurry of interest in subtyping for
intersection types that has led to several systems that satisfy the
subformula property. We discuss each of them in the next few
paragraphs.

\citet{Laurent:2018aa} introduces the ISC sequent-style system,
written $\Gamma \vdash B$, where $\Gamma$ is a sequence of types
$A_1,\ldots,A_n$. The intuition is that $A_1,\ldots,A_n \vdash B$
corresponds to $A_1 \cap \cdots \cap A_n \leq B$. The ISC system satisfies the
subformula property and is equivalent to the BCD system. To prove
this, Laurent establishes six lemmas that culminate in
cut-elimination, from which transitivity follows.

\citet{Bi:2018aa} present a subtyping algorithm that satisfies the
subformula property and prove that is equivalent to BCD
subtyping. Their system is based on the decision procedure
of~\citet{Pierce:1989aa} and takes the form $\mathcal{L} \vdash A
\prec: B$, where $\mathcal{L}$ is a queue of types that are peeled off
from the domain of the type on the right. So the judgement
$\mathcal{L} \vdash A \prec: B$ is equivalent to $A \leq \mathcal{L}
\to B$ where $\mathcal{L} \to B$ is defined as
\begin{align*}
  [] \to B &= B\\
  (\mathcal{L},A) \to B &= \mathcal{L} \to (A \to B)
\end{align*}
The proof of transitivity for this system is an adaptation of Pierce's
but it corrects some errors and introduces more lemmas concerning an
auxilliary notion called reflexive supertypes.

\citet{Muehlboeck:2018aa} develop a framework for obtaining subtyping
algorithms for systems that include intersection and union types and
that are extensible to other types. To decide $A \leq B$ their
algorithm converts $A$ to disjunctive normal form and then applies a
client-supplied function to each collection of literals. To obtain a
system equivalent to BCD, including the ${\to}{\cap}$ rule, the client
side function saturates the collection of literals by applying
${\to}{\cap}$ (left to right) as much as possible.
[todo: say something about the proof of transitivity]

This article presents a direct route to the subformula property and
proof of transitivity for BCD subtyping. We present straightforward
rules for a new subtyping relation $A <: B$ and directly prove
transitivity. The key to $A <: B$ is a rule for function types similar
to the one in Laurent's ISC, which effectively turns the
$\beta$-soundness property~\citep{Barendregt:2013aa} into a subtyping
rule. While the rules that define $A <: B$ are not particularly novel,
the fact that transivity can be proved directly from them is
suprising.

We describe the new subtyping relation in
Section~\ref{sec:new-subtyping}, prove transitivity in
Section~\ref{sec:trans}, and prove its equivalence to BCD subtyping in
Section~\ref{sec:equiv}. We make some concluding remarks in
Section~\ref{sec:conclude}.

The definitions and results in this article have been machine checked
in Agda.
    
%$λ^{\mathrm{BCD}}_\cap$

\section{A New Subtyping Judgment}
\label{sec:new-subtyping}

Our new subtyping judgment relies on a few definitions that are given
in Figure~\ref{fig:aux}. These include the partial functions $\dom{A}$
and $\cod{A}$, the $\topP{A}$ and $\topInCod{A}$ predicates, and the
relations $A \in B$ and $A \subseteq B$.
%
The $\dom{A}$ and $\cod{A}$ partial functions return the domain or
codomain if $A$ is a function type, respectively. If $A$ is an
intersection $A_1 \cap A_2$, then $\dom{A}$ is the intersection of the
domain of $A_1$ and $A_2$.  If $A$ is an atom, $\dom{A}$ is
undefined. Likewise for $\cod{A}$. For example, if $A = (A_1 \to B_1)
\cap \cdots \cap (A_n \to B_n)$, then $\dom{A} = A_1 \cap \cdots \cap
A_n$ and $\cod{A} = B_1 \cap \cdots \cap B_n$.  When $\dom{A}$ or
$\cod{A}$ appears in lemma or theorem statement, we implicitly assume
that $A$ is a type such that $\dom{A}$ and $\cod{A}$ are defined.
%
The $\topP{A}$ predicate identifies types that are equivalent to
$\TOP$. The $\topInCod{A}$ predicate identifies types that have $\TOP$
in their codomain.
%
The relation $A \in B$ indicates whether $A$ is syntactically a part of $B$.
The relation $A \subseteq B$ holds when every part of $A$ is a part of $B$.
We say that $B$ \emph{contains} $A$ if $A \subseteq B$.

\begin{proposition}\label{prop:union-subset-inv}
 \item If $A \cap B \subseteq C$, then $A \subseteq C$ and $B \subseteq C$. 
\end{proposition}


\begin{figure}[tbp]

  \fbox{$\dom{A}, \cod{A}$}
  \begin{align*}
  \dom{A \to B} &= A \\
  \dom{A \cap B} &= \dom{A} \cap \dom {B} \\
  \\
  \cod{A \to B} &= B \\
  \cod{A \cap B} &= \cod{A} \cap \cod {B}
  \end{align*}

  \fbox{$\topP{A}$}
  \begin{gather*}
    \inference{}{\topP{\TOP}}
    \quad
    \inference{\topP{B}}{\topP{A \to B}}
    \quad
    \inference{\topP{A} & \topP{B}}{\topP{A \cap B}}
  \end{gather*}

  \fbox{$A \in B$}
  \begin{gather*}
    \inference{}{\alpha \in \alpha}  \quad
    \inference{}{A \to B \in A \to B} \quad
    \inference{A \in B}{A \in B \cap C} \quad
    \inference{A \in C}{A \in B \cap C}
  \end{gather*}

  \fbox{$A \subseteq B$}
  \[
     A \subseteq B = ∀ C.\, C \in A \text{ implies } C \in B
  \]

  \fbox{$\mathsf{topInCod}(D)$}
  \[
  \mathsf{topInCod}(D) =
     \exists A B.\, A \to B \in D \text{ and } \mathsf{top}(B)  
  \]

  \caption{Auxiliary Definitions}
  \label{fig:aux}
\end{figure}


\begin{figure}[tbp]
  \fbox{$A <: B$}
  \begin{gather*}
    \text{(refl$_\alpha$)} \; \inference{}{\alpha <: \alpha} \\[3ex]
    \text{(lb$_L$)} \; \inference{A <: C}{A \cap B <: C} \quad
    \text{(lb$_R$)} \; \inference{B <: C}{A \cap B <: C} \quad
    \text{(glb)} \; \inference{A <: C & A <: D}{A <: C \cap D} \\[3ex]
    (\to') \; \inference{C <: \dom{B} & \cod{B} <: D }{A <: C \to D}
    \begin{array}{l} \neg\, \mathsf{top}(D) \\
      \neg \, \mathsf{topInCod}(B) \\
      B \subseteq A \end{array}\\[3ex]
    (\TOP_{\mathrm{top}}) \; \inference{}{A <: \TOP} \quad
    (\TOP{\to}') \; \inference{}{A <: C \to D}\;\mathsf{top}(D)
  \end{gather*}
  \caption{The New Subtyping Judgment}
  \label{fig:new-subtyping}
\end{figure}


The new intersection subtyping judgment, $A <: B$, is defined in
Figure~\ref{fig:new-subtyping}. First, it does not include the (trans)
rule.  It also replaces the (refl) rule with reflexivity for atoms
(refl$_\alpha$). The most important rule is the one for function types
$(\to')$, which subsumes $(\to)$ and $({\to}{\cap})$ in BCD subtyping.  The
$(\to')$ rule essentially turns the $\beta$-soundness property into a
subtyping rule. The $(\to')$ rule says that a type $A$ is a subtype of a
function type $C \to D$ if a subset of $A$, call it $B$, has domain and
codomain that are larger and smaller than $C$ and $D$,
respectively. The use of a subset of $A$ enables this rule to absorb
uses of (incl$_L$) and (incl$_R$) on the left.  The side conditions
$\neg\;\topP{B}$ and $\neg\;\topInCod{D}$ are needed because of the
$(\TOP{\to}')$ rule, which in turn is needed to preserve types under
$\eta$-reduction.  In a system that does not involve $\eta$-reduction,
the $(\TOP{\to}')$ rule can be omitted, as well as those side
conditions. The rules (lb$_L$) and (lb$_R$) adapt (incl$_L$) and
(incl$_R$) to a system without transitivity, and have appeared many
times in the literature~\citep{Bakel:1995aa}.  The $(\TOP{\to}')$ rule
generalizes the $(\TOP{\to})$ rule, replacing the $\TOP$ on the left with
any type $A$, because for transitivity, any type is below $\TOP$. The
$(\TOP{\to}')$ rule also replaces the $\TOP$ in the codomain on the
right with any type $D$ that is equivalent to $\TOP$.

Before moving on, we make note of some basic facts regarding the $<:$
relation and the $\topP{A}$ predicate.

\begin{proposition}[Basic Properties of $<:$]\ \label{prop:subtyping}
  \begin{enumerate}
  \item (reflexivity) $A <: A$ \label{prop:⊑-refl}
  \item If $A <: B \cap C$, then $A <: B$ and $A <: C$. \label{prop:⊔⊑-inv}
  \item If $A <: B$ and $C \in B$, then $A <: C$.\label{prop:in-sub-sub}
  \item If $A <: B$ and $C \subseteq B$, then $A <: C$.\label{prop:subset-sub-sub}
  \end{enumerate}
\end{proposition}
\begin{proof}\ 
  \begin{enumerate}
  \item The proof of reflexivity is by induction on $A$. In the case
    $A = A_1 \to A_2$, we proceed by cases on whether $\topP{A_2}$.
    If it is, deduce $A_1 \to A_2 <: A_1 \to A_2$ by rule $(\TOP{\to}')$.
    Otherwise, apply rule $(\to')$
  \item The proof is by induction on the derivation of $A <: B \cap C$.
  \item The proof is by induction on $B$. In the case where $B = B_1
    \cap B_2$, either $C \in B_1$ or $C \in B_2$, but in either case part
    \ref{prop:⊔⊑-inv} of this proposition fulfills the premise of
    the induction hypothesis, from which the conclusion follows.
  \item The proof  is by induction on $C$, using part \ref{prop:in-sub-sub}
    of this proposition in the cases for atoms and function types.
  \end{enumerate}
\end{proof}

\begin{proposition}[Properties of $\topP{A}$]\ \label{prop:top}
  \begin{enumerate}
  \item If $\topP{A}$ then $\topP{\cod{A}}$.\label{prop:AllBot-cod}
  \item If $\topP{A}$ and $B \in A$, then $\topP{B}$.\label{prop:AllBot-in}
  \item If $\topP{A}$ and $B \subseteq A$, then $\topP{B}$.\label{prop:AllBot-subset}
  \item If $\topP{A}$ and $A <: B$, then $\topP{B}$.\label{prop:AllBot-⊑}
  \item If $\topP{A}$, then $B <: A$.\label{prop:AllBot-⊑-any}
  \end{enumerate}
\end{proposition}
\begin{proof}\
  \begin{enumerate}
  \item The proof is a straightforward induction on $A$.
  \item The proof is also a straightforward induction on $A$.
  \item The proof is by induction on $B$. The cases for atoms and
    function types are proved by part \ref{prop:AllBot-in} of
    this proposition. In the case for
    $B = B_1 \cap B_2$, from $B_1 \cap B_2 <: A$, we have
    $B_1 <: A$ and $B_2 <: A$ (Proposition~\ref{prop:union-subset-inv}).
    Then by the induction hypotheses for $B_1$ and $B_2$ we have
    $\topP{B_1}$ and $\topP{B_2}$, from which we conclude
    that $\topP{B_1 \cap B_2}$.
  \item The proof is by induction on the derivation of $A <: B$.
    All of the cases are straightforward except for rule $(\to')$.
    In that case we have $B = B_1 \to B_2$ and some $A'$ such that
    $A' \subseteq A$, $B_1 <: \dom{A'}$, $\cod{A'} <: B_2$, $\neg\,\topP{B_2}$,
    and $\neg\,\topInCod{A'}$. From the premise $\topP{A}$
    and part \ref{prop:AllBot-subset} of this proposition, we
    have $\topP{A'}$. Then by part \ref{prop:AllBot-cod}
    we have $\topP{\cod{A'}}$. By the induction hypothesis
    for $\cod{A'} <: B$ we conclude that $\topP{B}$.
  \item The proof is a straightforward induction on $A$.
  \end{enumerate}
\end{proof}

Next we turn to the subtyping inversion principle for function types.
The idea is to generalize the rule $(\to')$ with respect to the type on
the right, allowing any type that contains a function type.  The
premises of $(\to')$ are somewhat complex, so we package most of them
into the following definition.

\begin{definition}[factors]
  We say $C \to D$ \emph{factors} $A$
  if there exists some type $B$ such that
  $B \subseteq A$, $C <: \dom{B}$, $\cod{B} <: D$, and
  $\neg\,\mathsf{topInCod}(B)$.
\end{definition}

\begin{proposition}[Inversion Principle for Function Types]
  \label{prop:⊑-fun-inv}
  If $A <: C'$, $C \to D \in C'$, and $\neg\,\mathsf{top}(D)$, then
  $C \to D$ factors $A$.
\end{proposition}
\begin{proof}
  The proof is a straightforward induction on $A <: C'$.
\end{proof}

\section{Transitivity}
\label{sec:trans}

The proof of transitivity relies on the following lemma, which is
traditionally needed to prove the inversion principle for function
types. However, it was not needed for our system because the rule
$(\to')$ is already quite close to the inversion principle.
The lemma states that if every function type $C \to D$ in A
factors $B$, then $\dom{A} \to \cod{A}$ also factors $B$.

\begin{lemma}\label{lem:sub-inv-trans}
  If
  \begin{itemize}
  \item for any $C$ $D$, if $C \to D \in A$ and $\neg\,\mathsf{top}(D)$,
    then $C \to D$ factors $B$, and
  \item $\neg\, \mathsf{topInCod}(A)$,
  \end{itemize}
  then $\dom{A} \to \cod{A}$ factors $B$.
\end{lemma}
\begin{proof}
  The proof is by induction on $A$.
  \begin{itemize}
  \item Case $A$ is an atom. The statement is vacuously true.
  \item Case $A = A_1 \to A_2$ is a function type. Then we conclude by applying
    the premise with $C$ and $D$ instantiated to $A_1$ and $A_2$ respectively.
  \item Case $A = A_1 \cap A_2$.  By the induction hypothesis for $A_1$
    and for $A_2$, we have that $\dom{A_1} \to \cod{A_1}$ factors $B$
    and so does $\dom{A_2} \to \cod{A_2}$.  So there exists
    $B_1$ and $B_2$ such that $B_1 \subseteq B$, $\neg\,\topInCod{B_1}$,
    $\dom{A_1} <: \dom{B_1}$, $\cod{B_1} <: \cod{A_2}$ and similarly
    for $B_2$. We need to show that $\dom{A} \to \cod{A}$ factors
    $B$. We choose the witness $B_1 \cap B_2$.  Clearly we have $B_1 \cap
    B_2 \subseteq B$ and $\neg \topInCod{B_1 \cap B_2}$.  Also, we have
    \[
    \dom{B_1} \cap \dom{B_2} <: \dom{A_1} \cap \dom{A_2}
    \]
    and
    \[
    \cod{A_1} \cap \cod{A_2} <: \cod{B_1} \cap \cod{B_2}
    \]
    Thus, we have that $\dom{B_1 \cap B_2} <: \dom{A}$
    and $\cod{A} <: \cod{B_1 \cap B_2}$, and this case is complete.
  \end{itemize}
\end{proof}

We now turn to the proof of transitivity, that if $A <: B$ and $B <:
C$, then $A <: C$. The proof is by well-founded induction on the
lexicographical ordering of the sum of the depths of $A$ and $C$
followed by the sum of the sizes of $B$ and $C$. To be precise, we
define this ordering as follows.
\[
\langle A, B, C \rangle \ll \langle A', B', C' \rangle
=
\begin{array}{l}
\depth{A} + \depth{C} < \depth{A'} + \depth{C'}
\text{ or } \\
(\depth{A} + \depth{C}
\leq
\depth{A'} + \depth{C'} \\
\text{ and }
\size{B} + \size{C}
<
\size{B'} + \size{C'})
\end{array}
\]
where $\size{A}$ is 
\begin{align*}
  \size{\alpha} &= 0 \\
  \size{A \to B} &= 1 + \size{A} + \size{B} \\
  \size{A \cap B} &= 1 + \size{A} + \size{B}
\end{align*}
and $\depth{A}$ is 
\begin{align*}
  \depth{\alpha} &= 0 \\
  \depth{A \to B} &= 1 + \max(\depth{A}, \depth{B}) \\
  \depth{A \cap B} &= \max(\depth{A}, \depth{B})
\end{align*}

\begin{theorem}[Transitivity of $<:$]\label{thm:⊑-trans}
    If $A <: B$ and $B <: C$, then $A <: C$.
\end{theorem}
\begin{proof}
  The proof is by well-founded induction on the relation $\ll$.
  We proceed by cases on the last rule applied in the
  derivation of $B <: C$.
  \begin{description}
  \item[Case (refl$_\alpha$)] We have $B = C = \alpha$.  From the premise $A <:
    B$ we immediately conclude that $A <: \alpha$.
  \item[Case (lb$_L$)] So $B = B_1 \cap B_2$, $B_1 <: C$, and $A <: B_1 \cap
    B_2$.  We have $A <: B_1$ (Proposition~\ref{prop:subtyping} part
    \ref{prop:⊔⊑-inv}), so we conclude that $A <: C$ by the induction
    hypothesis, noting that $\langle A, B_1, C \rangle \ll \langle A,
    B, C \rangle$ because $\size{B_1} < \size{B}$.
  \item[Case (lb$_R$)] So $B = B_1 \cap B_2$, $B_2 <: C$, and $A <: B_1 \cap
    B_2$.  We have $A <: B_2$ (Proposition~\ref{prop:subtyping} part
    \ref{prop:⊔⊑-inv}), so we conclude that $A <: C$ by the induction
    hypothesis, noting that $\langle A, B_2, C \rangle \ll \langle A,
    B, C \rangle$ because $\depth{B_2} \leq \depth{B}$ and $\size{B_2}
    < \size{B}$.
  \item[Case (glb)] We have $C = C_1 \cap C_2$, $B <: C_1$, and $B <:
    C_2$.  By the induction hypothesis, we have $A <: C_1$ and $A <:
    C_2$, noting that $\langle A, B, C_1 \rangle \ll \langle A, B, C
    \rangle$ $\langle A, B, C_2 \rangle \ll \langle A, B, C\rangle$
    because $\depth{C_1} \leq \depth{C}$, $\depth{C_2} \leq
    \depth{C}$, $\size{C_1} < C$, and $\size{C_2} < C$.
    We conclude $A <: C_1 \cap C_2$ by rule (glb).
  \item[Case $(\to')$] So $C = C_1 \to C_2$, $\neg \topP{C_2}$, and
    there exists $B'$ such that $C_1 <: \dom{B'}$, $\cod{B'} <: C_2$,
    $B' \subseteq B$, and $\neg \topInCod{B'}$. From $A <: B$ and $B' \subseteq B$, we
    have $A <: B'$ (Proposition~\ref{prop:subtyping} part
    \ref{prop:subset-sub-sub}). Thus, for any $B_1 \to B_2 \in B'$,
    $B_1 \to B_2$ factors $A$ (Proposition~\ref{prop:⊑-fun-inv}). We
    have satisfied the premises of Lemma~\ref{lem:sub-inv-trans}, so
    $\dom{B'} \to \cod{B'}$ factors $A$. That means there exists $A'$
    such that $A' \subseteq A$, $\neg\,\topInCod{A'}$, $\dom{B'} <: \dom{A'}$,
    and $\cod{A'} <: \cod{B'}$. Then by the induction hypothesis, we
    have
    \[
    C_1 <: \dom{A'} \quad\text{and}\quad \cod{A'} <: C_2
    \]
    noting that
    $\langle C_1, \dom{B'}, \dom{A'} \rangle \ll \langle A, B, C \rangle$
    because $\depth{C_1} + \depth{\dom{A'}} < \depth{A} + \depth{C}$
    and
    $\langle \cod{A'}, \cod{B'}, C_2 \rangle \ll \langle A, B, C \rangle$
    because $\depth{\cod{A'}} + \depth{C_2} < \depth{A} + \depth{C}$.
    We conclude that $A <: C_1 \to C_2$ by rule $(\to')$ witnessed by $A'$.
  \item[Case $(\TOP_{\mathrm{top}})$] We have $C = \TOP$ and
    conclude $A <: \TOP$ by rule $(\TOP_{\mathrm{top}})$.
  \item[Case $(\TOP{\to}')$] We have $C = C_1 \to C_2$ and $\topP{C_2}$.
    We conclude $A <: C_1 \to C_2$ by rule $(\TOP{\to}')$.
  \end{description}
\end{proof}


\section{Equivalence with BCD Subtyping}
\label{sec:equiv}

Having proved (trans), we next prove $(\to)$ and $({\to}{\cap})$
and then show that $A <: B$ is equivalent to $A \leq B$.

\begin{lemma}[$\to$]\label{lem:⊑-fun′}
  If $C <: A$ and $B <: D$, then
  $A \to B <: C \to D$.
\end{lemma}
\begin{proof}
  Consider whether $\topP{D}$ or not.
  \begin{description}
  \item[Case $\topP{D}$] We conclude $A \to B <: C \to D$
    by rule $(\TOP{\to}')$.
  \item[Case $\neg \,\topP{D}$]
    Consider whether $\topP{B}$ or not.
    \begin{description}
    \item[Case $\topP{B}$] So $\topP{D}$ (Prop. \ref{prop:top}
      part \ref{prop:AllBot-⊑}), but that is a contradiction.
    \item[Case $\neg\,\topP{B}$]
      We conclude that $A \to B <: C \to D$ by rule $(\to')$.
    \end{description}
  \end{description}
\end{proof}

\begin{lemma}[${\to}{\cap}$]\label{lem:⊑-dist}
  $(A \to B) \cap (A \to C) <: A \to (B \cap C)$
\end{lemma}
\begin{proof}
   We consider the cases for whether $\topP{B}$ or $\topP{C}$.
   \begin{description}
   \item[Case $\topP{B}$ and $\topP{C}$]
     Then $\topP{B \cap C}$ and we conclude that
     $(A \to B) \cap (A \to C) <: A \to (B \cap C)$
     by rule $(\TOP{\to}')$.
   \item[Case $\topP{B}$ and $\neg\,\topP{C}$]
     We conclude that 
     $(A \to B) \cap (A \to C) <: A \to (B \cap C)$
     by rule $(\to')$, choosing the witness $A \to C$
     and noting that $C <: B$ 
     by way of Proposition~\ref{prop:top}
     part \ref{prop:AllBot-⊑-any}
     and $C <: C$ by Proposition~\ref{prop:subtyping}
     part \ref{prop:⊑-refl}.
   \item[Case $\neg\,\topP{B}$ and $\topP{C}$]
     We conclude that 
     $(A \to B) \cap (A \to C) <: A \to (B \cap C)$
     by rule $(\to')$, this time with witness $A \to B$
     and noting that
     $B <: B$ by Proposition~\ref{prop:subtyping}
     part \ref{prop:⊑-refl}
     and $B <: C$ 
     by way of Proposition~\ref{prop:top}
     part \ref{prop:AllBot-⊑-any}.
   \item[Case $\neg,\topP{B}$ and $\neg\,\topP{C}$]
     Again we apply rule $(\to')$, but with witness
     $(A \to B) \cap (A \to C)$.
   \end{description}
\end{proof}

\noindent We require one more lemma.

\begin{lemma}
  \label{lem:dv↦cv<:v}
  $A \leq \dom{A} \to \cod{A}$.
\end{lemma}
\begin{proof}
  The proof is by induction on $A$.
\end{proof}

\noindent Now for the proof of equivalence

\begin{theorem}[Equivalence of the subtyping relations]\ \\
  $A <: B$ if and only if $A \leq B$.
\end{theorem}
\begin{proof}
  We prove each direction of the if-and-only-if separately.
  \begin{description}
  \item[$A <: B$ implies $A \leq B$]
    We proceed by induction on the derivation of $A <: B$.
    \begin{description}
    \item[Case (refl$_\alpha$)] We conclude $\alpha \leq \alpha$ by (refl).      
    \item[Case (lb$_L$)] By the induction hypothesis we have $A \leq C$.
      By (incl$_L$) we have $A \cap B \leq A$. We conclude
      that $A \cap B \leq C$ by (trans).
    \item[Case (lb$_R$)] By the induction hypothesis we have $B \leq C$.
      By (incl$_R$) we have $A \cap B \leq B$. We conclude
      that $A \cap B \leq C$ by (trans).
    \item[Case (glb)] By the induction hypothesis we have
      $A \leq C$ and $A \leq D$, so we conclude that
      $A \leq C \cap D$ by (glb).
    \item[Case $(\to')$] By the induction hypothesis we have
      $C \leq \dom{B}$ and also $\cod{B} \leq D$.
      From $B \subseteq A$ we have $A \leq B$.      
      Then by Lemma~\ref{lem:dv↦cv<:v} we have
      $B \leq \dom{B} \to \cod{B}$.
      Also, we have $\dom{B} \to \cod{B} \leq C \to D$ by rule $(\to)$.
      We conclude that $A \leq C \to D$ by chaining the three prior
      facts using (trans).
    \item[Case $(\TOP_{\mathrm{top}})$]
      We conclude that $A \leq \TOP$ by $(\TOP_{\mathrm{top}})$.
    \item[Case $(\TOP{\to}')$] We have $A \leq \TOP$ and $\TOP <: C \to
      \TOP$.  Also, $C \to \TOP \leq C \to D$ because $\TOP <: D$ follows
      from $\topP{D}$.  Thus, applying (trans) we conclude $A \leq C \to
      D$.
    \end{description}
    
  \item[$A \leq B$ implies $A <: B$]
    We proceed by induction on the derivation of $A \leq B$.
    \begin{description}
    \item[Case (refl)] We conclude $A <: A$ by Prop.~\ref{prop:subtyping}
      part \ref{prop:⊑-refl}.
    \item[Case (trans)]
      By the induction hypothesis, we have $A <: B$ and $B <: C$.
      We conclude that $A <: C$ by Theorem~\ref{thm:⊑-trans}.
    \item[Case (incl$_L$)] We have $A <: A$
      (Prop.~\ref{prop:subtyping} part \ref{prop:⊑-refl}),
      and therefore $A \cap B <: A$ by rule (lb$_L$).
    \item[Case (incl$_R$)] We have $B <: B$
      (Prop.~\ref{prop:subtyping} part \ref{prop:⊑-refl}),
      and therefore $A \cap B <: B$ by rule (lb$_R$).
    \item[Case (glb)] By the induction hypothesis, we have
      $A <: C$ and $A <: D$, so we conclude that $A <: C \cap D$ by (glb).
    \item[Case $(\to)$] By the induction hypothesis, we have $C <: A$
      and $B <: D$. We conclude that $A \to B <: C \to D$
      by Lemma~\ref{lem:⊑-fun′}.
    \item[Case $({\to}{\cap})$] We conclude that
      $(A \to B) \cap (A \to C) <: A \to (B \cap C)$
      by Lemma~\ref{lem:⊑-dist}.
    \item[Case $(\TOP_{\mathrm{top}})$]
      We conclude that $A <: \TOP$ by rule $(\TOP_{\mathrm{top}})$.
    \item[Case $(\TOP{\to})$]
      We have $\topP{\TOP}$, so $\TOP <: C \to \TOP$ by rule $(\TOP{\to}')$.
    \end{description}
    
  \end{description}
\end{proof}

\section{Conclusion}
\label{sec:conclude}

In this article we present a new subtyping relation $A <: B$ for
intersection types that enjoys the subformula property.  None of the
rules of the new subtyping relation are particularly novel, but the
fact that we can prove transitivity directly from them is!  We prove
that the new relation is equivalent to the subtyping relation $A \leq B$
of Barendregt, Coppo, and Dezani-Ciancaglini.


\section*{Acknowledgments}

This material is based upon work supported by the National Science
Foundation under Grant No. 1814460.

\pagebreak

\bibliographystyle{abbrvnat}
\bibliography{all}

\end{document}

% LocalWords:  dom subtyping Barendregt et al subformula sequent aa
% LocalWords:  BCD lclr refl glb topInCod Coppo Dezani Ciancaglini fk
% LocalWords:  Salle Engeler Honsell Abramsky Plotkin Rocca Alessi
% LocalWords:  subsumption subtype denotational ISWIM Landin Ishihara
% LocalWords:  Felleisen Muehlboeck ISC Agda codomain reflexivity
% LocalWords:  Bakel
